\documentclass{article}

\usepackage[utf8]{inputenc}

\usepackage{amsmath}
\usepackage{amssymb}
\usepackage{mathtools}  % for \coloneqq
\usepackage{amsthm}
\usepackage{microtype}
\usepackage[english]{babel}
\usepackage{csquotes} % for quotes in citations

% \usepackage{biblatex}
% \addbibresource{ref.bib}
\newcommand{\ds}{\mathop{ds}}
\newcommand{\dx}{\mathop{dx}}
\newcommand{\dt}{\mathop{dt}}
\newcommand{\dxdt}{\mathop{dx}\mathop{dt}}
\newcommand{\dsdt}{\mathop{ds}\mathop{dt}}

\newcommand{\R}{\mathbb{R}}
\newcommand{\placeholder}{\makebox[1ex]{\textbf{$\cdot$}}}

\usepackage[T1]{fontenc}
\usepackage{microtype}
\usepackage{microtype}
\usepackage{lmodern}
\usepackage{graphicx}
\usepackage{titling}
\usepackage{titlesec}
\usepackage{environ}
\usepackage{xcolor}
\usepackage{tikz}

\definecolor{dark}{HTML}{3E454C}
\definecolor{blue}{HTML}{2185C5}
\definecolor{teal}{HTML}{7ECEFD}
\definecolor{pinkish}{HTML}{FF7F66}
\definecolor{red}{HTML}{D90000}

% Font sizes:    Calculated using 1.2^k, k=1,...,5
% 1.2
% 1.44
% 1.728
% 2.0736
% 2.48832

\newtheoremstyle{pretty}
  {\topsep}   % ABOVESPACE
  {\topsep}   % BELOWSPACE
  {\itshape}  % BODYFONT
  {0pt}       % INDENT (empty value is the same as 0pt)
  {\fontseries{b}\selectfont} % HEADFONT
  {.}         % HEADPUNCT
  {5pt plus 1pt minus 1pt} % HEADSPACE
  {}          % CUSTOM-HEAD-SPEC

\theoremstyle{pretty}



\pretitle{
    \begin{center}
    \fontsize{1.728em}{1em}\fontseries{b}\selectfont
}
\posttitle{
    \par
    \end{center}
    \vskip 0.5em
}
\preauthor{
    \begin{center}
    \fontsize{1.2em}{1.44em}\selectfont
}
\postauthor{
    \end{center}
}
\predate{
    \begin{center}
    \fontsize{1.2em}{1.44em}\selectfont
}
\postdate{
    \end{center}
}

\titleformat*{\section}{\fontsize{14.4pt}{1.728pt}\fontseries{b}\selectfont}
\titleformat*{\subsection}{\fontsize{1.2em}{1.44em}\fontseries{b}\selectfont}
\titleformat*{\subsubsection}{\fontseries{b}\selectfont}

\newcommand{\textb}[1]{{\fontseries{b}\selectfont #1}}

\NewEnviron{myBox}[1][teal!40]
{
\begin{center}
\begin{tikzpicture}
    \node [fill=#1, text width=\linewidth-1cm, align=justify, inner xsep=5mm, inner ysep=3mm] at (0, 0)
    {\BODY};
\end{tikzpicture}
\end{center}
}
\renewcommand{\L}{\mathcal{L}}
\addbibresource{ref.bib}

\newtheorem{theorem}{Theorem}[section]
\newtheorem{corollary}{Corollary}[theorem]
\newtheorem{lemma}[theorem]{Lemma}


\title{TMA4183 Steel Optimization}
\author{}
\date{Spring 2020}

\begin{document}

\maketitle

\section{Introduction}
We want to consider the optimal control problem for the controlled cooling of steel in order to obtain a desired temperature, $\theta_d$. In order to model such a cooling process mathematically we use linear heat equation as state equation given by
\begin{subequations}
   \label{eq:heat}
   \begin{align}
      \rho c_p \theta_t - \nabla \cdot (k \nabla \theta) &= 0 \quad &\text{in } \Omega \times (0,T),\label{eq:heat-in-omega} \\
      -k \frac{\partial \theta}{\partial \nu} &= u(t) (\theta - \theta_w) \quad &\text{on } \Gamma_1 \times (0,T), \\
      -k \frac{\partial \theta}{\partial \nu} &= 0 \quad &\text{on } \Gamma_0 \times (0,T), \\
      \theta(x, 0) &= \theta_0 &\text{in } \Omega &
   \end{align}
\end{subequations}
Where $u = u(t) \colon \mathbb{R} \to \mathbb{R}$ is the scalar-valued, time dependent heat transfer coefficient which acts as our unknown control. $c_p$ is the heat capacity, $k$ is thermal conductivity and $\rho$ is the density. The Neumann boundary condition in (1b) is based on Newton's law of cooling, here $\theta_w$ is the temperature of the applied coolant at the boundary. The goal is to determine a optimal $u$ such that $\theta(x, t_f)$ is "as close as possible" to some desired temperature $\theta_d$, where $t_f$ denotes the final time of the process, and we furthermore require that $u \in U_{\mathrm{ad}}$. More formally, we wish to minimize a cost functional of the form
\begin{subequations}
\begin{equation}\label{eq:cost-func}  % Abuse of notation?
   \min J(\theta, u) = \frac{1}{2} \int_\Omega (\theta(x, T) - \theta_d)^2 \mathop{dx} \mathop{dt} + \frac{\gamma}{2} \int_0^{T} u^2 \mathop{dt}
\end{equation}
subject to
\begin{equation}
      \theta \colon \Omega \times [0, T] \to \mathbb{R} \text{ solves \eqref{eq:heat}},
\end{equation}
and
\begin{equation}
   u \in U_{\mathrm{ad}}.
\end{equation}
\end{subequations}
We have defined $Q \coloneqq \Omega \times (0, T)$. Here $\gamma \geq 0$ is a penalizing parameter called Tikhonov regularization parameter, and the last term in $J(\theta, u)$ is a regularizing term which penalize high costs of the control. 

\section{Main part}

\subsection{Weak formulation and the adjoint eqution}

We start by finding the weak formulation for the problem defined in \eqref{eq:heat}. We multiply the equation \eqref{eq:heat-in-omega} by a test function $\psi\in H^1(\Omega)$, and integrate over the domain $Q$. Then partial integration gives
\begin{equation}
\begin{aligned}
  0 &= \iint_Q (\theta_t - \frac{k}{\rho c_p}\Delta\theta)\psi\dxdt  \\
  &= \iint_Q \theta_t\psi\dxdt + \frac{k}{\rho c_p}\iint_Q\nabla\theta\nabla\psi\dxdt - \frac{k}{\rho c_p}\iint_{\partial Q}\partial_\nu\theta\psi\dxdt.
\end{aligned}
\end{equation}
Now after inserting for the boundary conditions defined in \eqref{eq:heat}, we end up with the following weak formulation of the state equation
\begin{equation}
  \begin{aligned}\label{eq:weak-form}
  \iint_Q \theta_t\psi\dxdt + \frac{k}{\rho c_p}\iint_Q\nabla\theta\nabla\psi &\dxdt + \frac{1}{\rho c_p}\iint_{\Sigma_1} u(t)\theta\psi\dsdt \\
  &= \frac{1}{\rho c_p}\iint_{\Sigma_1} u(t)\theta_w\psi\dsdt,
  \end{aligned}
\end{equation}
where $\Sigma_i = \Gamma_i\times(0,T)$.

From the weak formulation of the problem, we can easily set up the Lagrangian function, by subtracting the weak formulation of the problem \eqref{eq:weak-form} from the cost functional \eqref{eq:cost-func}. The test function is replaced with a Lagrange multiplier function $p$. This gives
\begin{equation}
  \begin{aligned}\label{eq:lagrangian-raw}
  \mathcal{L}(\theta, u, p) = &\,J(\theta, u) - \iint_Q \theta_t p\dxdt - \frac{k}{\rho c_p}\iint_Q\nabla\theta\nabla p \dxdt \\
  &- \frac{1}{\rho c_p}\iint_{\Sigma_1} u(t)\theta p\dsdt
  + \frac{1}{\rho c_p}\iint_{\Sigma_1} u(t)\theta_w p\dsdt.
  \end{aligned}
\end{equation}

\subsubsection{Adjoint equation}
Now we derive the adjoint equation. This is done by  setting the directional derivative of the Lagrangian (with respect to the state variable) equal to zero. Assume the direction $h\in H^1(Q)$ is such that $h(x, 0) = 0$. % Antagelsen h(x, 0) = 0 popper visst ut av seg selv hvis man gjør noe på en bestemt måte i følge Dietmar.
This gives
\begin{equation}
  \begin{aligned}
  0 = \mathcal{L}_\theta(\theta, u, p)h = \int_\Omega (\theta(x,T) - \theta_d)h(x, T)\dx - \iint_Q h_t p\dxdt \\
  - \frac{k}{\rho c_p}\iint_Q\nabla h\nabla p \dxdt
  - \frac{1}{\rho c_p}\iint_{\Sigma_1} u(t)h p\dsdt
  \end{aligned}
\end{equation}
Partial integration then gives
\begin{equation} % Skip this calculation?
  \begin{aligned}
  0 = \mathcal{L}_\theta(\theta, u, p)h = \int_\Omega \theta(x,T) - \theta_d)h(x, T)\dx - \int_\Omega hp\Big|_0^T \dx \\
  + \iint_Q h p_t\dxdt
  + \frac{k}{\rho c_p}\iint_Q h\Delta p \dxdt \\
  - \frac{k}{\rho c_p} \iint_{\partial Q}\partial_\nu p\cdot h\dsdt
  - \frac{1}{\rho c_p}\iint_{\Sigma_1} u(t)h p\dsdt
  \end{aligned}
\end{equation}
Now after som reordering of the terms we get
\begin{equation}
  \begin{aligned}
  0 = \mathcal{L}_\theta(\theta, u, p)h = \int_\Omega \theta(x,T) - \theta_d-p(x, T))h(x, T)\dx \\
  + \iint_Q h \left( p_t + \frac{k}{\rho c_p}\Delta p\right) \dxdt
   - \frac{k}{\rho c_p} \iint_{\Sigma_2}\partial_\nu p\cdot h\dsdt \\
   + \frac{1}{\rho c_p} \iint_{\Sigma_1}h(  -k\partial_\nu p - u(t)p )\dsdt.
  \end{aligned}
\end{equation}
Now we let $h\in H_0^1(Q)$. Then we are only left with the second integral, which must still be equal to zero for every $h$ in the chosen function space. We then get that
\begin{equation*}
  \rho c_p p_t + k\Delta p = 0 \quad\textrm{ in } \Omega.
\end{equation*}
Now letting $h\in H^1(Q)$ such that $h(x, T) = 0$ and $h|_{\Sigma_2}=0$ gives that
\begin{equation*}
  -k\frac{\partial p}{\partial\nu} = u(t)p \quad\textrm{ in } \Sigma_1.
\end{equation*}
Now we replace tha last condition on $h$ from above with $h|_{\Sigma_1}=0$ and get
\begin{equation*}
  -k\frac{\partial p}{\partial\nu} = 0 \quad\textrm{ in } \Sigma_2.
\end{equation*}
Now finally we just let $h\in H^1(Q)$ and we are left with
\begin{equation*}
  p(x, T) = \theta(x, T) - \theta_d
\end{equation*}
This constitues the \textit{adjoint equation}, which we restate below
\begin{subequations}
   \label{eq:adjoint-eqn}
   \begin{align} % Approved by Dietmar!
      \rho c_p p_t + k\Delta p &= 0 \quad\qquad\textrm{ in } \Omega \\
      -k\frac{\partial p}{\partial\nu} &= u(t)p \,\,\quad\textrm{ in } \Sigma_1 \\
      -k\frac{\partial p}{\partial\nu} &= 0 \,\quad\qquad\textrm{ in } \Sigma_2 \\
      p(x, T) &= \theta(x, T) - \theta_d.
   \end{align}
\end{subequations}

\section{Existence and uniquenes for the optimal control\\\mbox{problem}}\label{proof}

\subsection{Existence and uniqueness state system}
We will now consider the existence of a solution to the state equation. In order to do so, we consider a more general linear parabolic partial differential equation, show a solution exists for this general form. We then use that result to conclude about the existence for our particular problem as well.

Consider the PDE given by
%
\begin{align}
    \label{eq:moreGen}
    u_t - Au + c_0 y = f \quad \text{in } Q = \Omega \times (0,T) \\
    \label{eq:moreGen2}\partial_{n_A} u + \alpha u = g \quad \text{on } \Sigma = \Gamma \times (0,T) \\
    u(\placeholder,0) = u_0(\placeholder) \quad \text{in } \Omega
\end{align}
It does not predescribe any difficulties by splitting the boundary $\Gamma$ into disjoint parts
and predescribe Neumann conditions on these sections separately, as we have done in \ref{eq:heat}. The functions $\alpha$, $\beta$, $f$ and $g$ all depend on $(x,t)$. Here $A$ is an elliptic differential operator of the following form:
%
\begin{equation*}
    Af(x) \coloneqq -\sum_{i,j}^N\frac{\partial}{\partial x_i}(a_{ij}(x)\frac{\partial}{\partial x_j}f(x)) \quad x\in \Omega
\end{equation*}
One need some requirements on the coefficents of the matrix $A=(a_{ij})$, that is one require $a_{ij} \in L^{\infty}(\Omega)$ and that they are symmetric, i.e. $a_{ji}(x) = a_{ij}(x)$ for $x\in \Omega$. Moreover, one also assumes that the uniform ellipticity condition is satisfied, that is there is some $M > 0$ such that 
\begin{equation*}
    \label{eq:uniformEl}
    \sum_{i,j}^N a_{ij}(x)x_i x_j \geq M|x|^2 \quad \forall x \in \mathbb{R}^N \text{ for a.e. $x\in \Omega$}
\end{equation*}
The operator $\partial_{n_A}$ is the directional derivative of the conormal vector $n_{A}$, a vector whose components are given by $n_{A} = An$, where A is the elliptic differential operator. We present a theorem stating the existence and uniqueness for a solution to \eqref{eq:moreGen}. 

\begin{theorem}[Existence and uniqueness] Assume that $\Omega \subset \mathbb{R}^n$ is a bounded Lipschitz domain with boundary $\Gamma$ and let $T>0$ denote the final time. Moreover, assume $c_0 \in L^{\infty}(Q)$ and $\alpha \in L^{\infty}(\Sigma)$, where $\alpha(x,t) \geq 0$ for a.e. $(x,t) \in \Sigma$, $y_0 \in L^2(\Sigma), f \in L^2(Q)$, and $g \in L^2(\Sigma)$. Then the parabolic initial-value problem \eqref{eq:moreGen} has a unique weak solution in $W_2^{1,0}(Q) \cap L^{\infty}(0,T;L^2(\Omega))$. Moreover, there is a constant $C>0$ which is independent of $f$, $g$ and $u_0$ such that 
\begin{equation*}
    \max_{t \in [0,T]} \|u(.,t)\|_{L^2(\Omega)} + \|u\|_{W_2^{1,0}(Q)} \leq C\left(\|f\|_{L^2(Q)} + \|g\|_{L^2(\Sigma)} + \|u_0\|_{L^2(\Omega)}\right).
\end{equation*}
This holds for all $f \in L^2(Q), g \in L^2(\Sigma)$ and $u_0 \in L^2(\Omega)$
\end{theorem}

\begin{proof}
The proof is given by Tröltzsch in \cite{optimalControl} under theorem 7.8. We will provide a rough sketch of the required steps in the proof. WOLOG one may assume that $c_0(x,t)\geq 0$ for a.e. $(x,t) \in Q$, since if that were not the case one could do the substitution $y(x,t) \rightarrow e^{\lambda t}\bar{y}(x,t)$. Then the resulting differential equation for $\bar{y}$ involves the term $(\lambda + c_0)\bar{y}$ instead of $c_0y$ as in the original PDE. Taking $\lambda >0$ large enough, this will certainly be positive. 

In order to prove the existence in the theorem one need to proceed in 4 steps, these are

\begin{enumerate}
    \item Make a Galerkin approximation to the problem,
    \item Estimate the sequence $\{y_n \}$ which is an approximation sequence for the states of the PDE
    \item Consider the convergence of the sequence of estimated controls $\{u_j^N\}_j$ and states $\{ y_n \}$
    \item Show that the limit $y_n \rightarrow y$ is indeed a weak solution to the PDE.
\end{enumerate}
To prove uniqueness, one has to use an energy inequality together with the regularity of the solution.
\end{proof}


From this theorem, we can make a conclusion regarding the existence of a solution to our particular parabolic PDE. Let $c_0 = 0$, $A = \frac{1}{\rho c_p}\nabla \cdot (k\nabla)$, and $f = 0$. Then $\partial_{n_A} = \partial_n$. We also set $\Gamma = \Gamma_1 \cup \Gamma_2$, to partition our boundary into to disjoint separate parts, now let $\alpha_1 = -\frac{u(t)}{k}$ and $g_1 = -\frac{u(t)\theta_w}{k}$, while one set $\alpha_2 = 0, g_2 = 0$, where $\alpha_i$ is the value of $\alpha$ on $\Sigma_i$ and similarly for $g_i$. This results in the next corollary. 

\begin{corollary}[Existence]
Suppose that $\theta_w \in L^{\infty}(\Sigma_1)$, $\theta_0 \in L^2(\bar{\Omega})$, $u \in L^{\infty}(0,T)$ and $u\geq 0$. Then the initial value \eqref{eq:heat} admits a unique solution $\theta \in W_2^{1,0}(Q)$. After a possible modification on a null set, we have $\theta \in W(0,T)$. This weak solution to the PDE satisfy an upper bound 
\begin{equation*}
    \|\theta \|_{W(0,T)} \leq C\bigg ( \|g_1\|_{L^2(\Sigma)} + \|\theta_0\|_{L^2(\Sigma)} \bigg )
\end{equation*}
for a constant $C>0$ independent on $(g_1, \theta_0)$
\end{corollary}

Actually using theorem 5.5 in \cite{optimalControl} we get that our solution 
\begin{equation*}
    \theta \in W(0,T) \cap C (\bar{Q})
\end{equation*}
and as we will see the solution to the adjoint system satisfy have the same regularity using a time transformation. 


\subsection{Existence and uniqueness Adjoint system}
Want to prove that there exist a unique weak solution $p \in W_2^{1,0}(Q)$ to our adjoint system, and modifying $p$ on a nullset we get $p\in W(0,t)$. We consider the sightly more general parabolic initial-boundary value problem 
\begin{align*}
    -p_t -\nabla^2p +c_0p = a_Q \qquad \text{in } Q, \\
    \partial_np + \alpha p = a_{\Sigma} \qquad \text{on } \Sigma, \\
    p|_{t=T} = a_{\Omega} \qquad \text{in } \Omega.
\end{align*}
If we assume bounded and measurable coefficient functions i.e. $c_0, \alpha, a_Q \in L^2(Q)$, $a_{\Sigma} \in L^2(\Sigma)$, and $a_{\Omega} \in L^2(\Omega)$, then we can introduce a bilinear form
\begin{equation}
    \label{eq:parabolic_adj}
    A(y,v)(t) := \int_{\Omega}(\nabla y \cdot \nabla v + c_0(\placeholder,t) y v) \dx + \int_{\Gamma}\alpha(\placeholder,t) y v \ds
\end{equation}
Then the stated PDE is well-posed according to the following Lemma:

\begin{lemma}[Well-posedness]
The parabolic adjoint system given in \eqref{eq:parabolic_adj} has a unique weak solution $p \in W_2^{1,0}(Q)$ which is a solution to the variational problem 
\begin{equation*}
    \iint_Q pv_t \dxdt + \int_0^TA(p,v)(t)\dt = \int_{\Omega}a_{\Omega}v(T) \dt + \iint_Q a_Qv \dxdt + \iint_{\Sigma}a_{\Sigma}v \dsdt
\end{equation*}
this should hold for all $v \in H^{1,1}(Q)$ with $v|_{t=0} = 0$. If we modify $p$ on a null set we have $p\in W(0,T)$, and there exists some constant $M>0$ that does not depend on $(a_Q,a_{\Sigma}, a_{\Omega})$ such that 
\begin{equation*}
    \|p\|_{W(0,T)} \leq M \bigg (\|a_Q\|_{L^2(Q)} + \|a_{\Sigma}\|_{L^2(\Sigma)} + \|a_{\Omega}\|_{L^2(\Omega)} \bigg ).
\end{equation*}
\end{lemma}

\begin{proof}
The idea is to reduce the system to a forward parabolic initial-boundary value problem. This is done with a time transformation, taking $\tau \in [0,T]$ and introducing
\begin{equation*}
    \hat{p}(\tau) := p(T-\tau).
\end{equation*}
A similar transform is applied with $\hat{v}(\tau)$. Then $\hat{p}(0) = p(T)$ and $\hat{p}(T) = p(0)$; analogously for $\hat{v}$. We have that $\hat{a_Q}(\placeholder,t):= a_Q(\placeholder,T-\tau)$, and likewise for all the other coefficients. Due to this time transform, we have that
\begin{equation*}
    \iint_Qpv_t \dxdt = - \iint_Q \hat{p}\hat{v_{\tau}} \dxdt.
\end{equation*}
Considering the weak formulation, it now corresponds to the forward parabolic initial-boundary value problem given by 
\begin{align*}
    \hat{p_{\tau}} - \nabla^2 \hat{p} + c_0 \hat{p} = \hat{a_Q} \qquad \text{in } Q \\
    \partial_n \hat{p} + \alpha \hat{p} = \hat{a_{\Sigma}} \qquad \text{on } \Sigma \\
    \hat{p}(0) = \hat{a_{\Omega}} \qquad \text{in } \Omega
\end{align*}
By the previous existence and uniqueness theorem, this problem admits a unique weak solution $\hat{p} \in W(0,T)$. Reversing the time transformation we have proven uniqueness and existence for the adjoint system.
\end{proof}
This system is more general than ours, so we can reduce it to our case. Consider \eqref{eq:adjoint-eqn}, if we set $c_0 = 0$, $a_Q = 0$ and split up the boundary $\Sigma = \Sigma_1 \cup \Sigma_2$ where we set $\alpha_1 = \frac{u(t)}{k}$ and $\alpha_2 = 0$, $a_{\Sigma_i}=0$ for $i\in \{1,2 \}$ and finally set $a_{\Omega} = \theta|_{t=T}-\theta_d$ we have reduced to the case of the lemma, consequently we have existence and uniqueness for \eqref{eq:adjoint-eqn}.



\subsection{Existence of optimal control}
Under the required conditions for a solution to \eqref{eq:heat} we can go on to prove the existence of an optimal control $\bar{u}$. 

\begin{lemma}[WLSC]
    The cost functional given b y
    \begin{equation*}
         J(\theta, u) = \frac{1}{2} \int_\Omega (\theta(x, T) - \theta_d)^2 \mathop{dx} + \frac{\gamma}{2} \int_0^{T} u^2 \mathop{dt}
    \end{equation*}
    is weakly lower semicontinous (WLSC). 
\end{lemma}

\begin{proof}
We can reformulate the cost functional $J(\theta, u)$ in terms of norms as 
\begin{equation*}
    J(\theta, u) = \frac{1}{2}\|\theta (,T) - \theta_d \|_{L^2(\Omega)}^2 + \frac{\gamma}{2}\|u\|_{L^2(0,T)}^2
\end{equation*}
Now a norm is convex by applying the triangle inequality and continuity of the norm follows by an application of the reverse triangle inequality. This thus implies that the cost functional is weakly lower semicontinuous in both arguments.
\end{proof}

\begin{theorem}[Optimal Control]
If the assumptions in the existence corollary are all satisfied, then there exist at least one solution to the optimal control problem 
\begin{equation*}
       \min J(\theta, u) = \frac{1}{2} \int_\Omega (\theta(x, T) - \theta_d)^2 \mathop{dx} \mathop{dt} + \frac{\gamma}{2} \int_0^{T} u^2 \mathop{dt}
\end{equation*}
subject to
\begin{align*}
       \rho c_p \theta_t - \nabla \cdot (k \nabla \theta) &= 0 &&\text{in } Q,  \\
      -k \frac{\partial \theta}{\partial \nu} &= u(t) (\theta - \theta_w) &&\text{on } \Sigma_1, \\
      -k \frac{\partial \theta}{\partial \nu} &= 0 \quad &&\text{on } \Sigma_2, \\
      \theta(x, 0) &= \theta_0 && \text{in $\Omega$.}
\end{align*}
\end{theorem}

\begin{proof}
We follow the same footsteps as in  \cite{DPSteel}. By the existence corollary for the parabolic PDE we know that  for any $u \in U_{ad} \text{ } \exists \text{ }\theta \in W(0,T)\cap C(\bar{Q})$. The set of admissible controls $U_{ad}$ is bounded in $L^{\infty}(0,T)$ therefore the solution $\theta$ is bounded in $W(0,T) \cap C(\bar{Q})$ thus the cost functional will be bounded. We can as a consequence assume we have a minimising sequence which we represent by the tuple $\{(\theta_k,u_k)\}_{k\in \mathbb{N}}$ such that we have 
\begin{equation*}
    \lim_{k\rightarrow \infty}J(\theta_k,u_k) = \inf_{(\theta,u)\in (W(0,T) \cap C(\bar{Q})\times U_{ad}}J(\theta,u)
\end{equation*}
where $(\theta_k,y_k) = S(u_k)$ where $S(\placeholder)$ is the solution operator which assigns the solution to the state system for a given control $u_k$.

Now $U_{ad}$ is a closed linear subspace of the Banach space $L^{\infty}(0,T)$, and is therefore itself a  Banach space, furthermore $U_{ad}$ is bounded and convex, so the sequence $\{u_k\}$ will be bounded, and there exist a weakly convergent subsequence of the controls $\{u_{k'} \} \subset \{u_k \}$ and a limit $\bar{u} \in L^{\infty}(0,T)$ such that
\begin{equation*}
    u_{k'} \rightharpoonup \bar{u} \text{ in } L^2(0,T).
\end{equation*}

Now since $U_{ad}$ is closed and convex, $U_{ad}$ is weakly closed, therefore $\bar{u} \in U_{ad}$. \bigskip

Using our existence corollary we know there exist a unique solution to the state system also with the control $\bar{u}$. Extracting a further subsequence if necessary still indexed by $k'$ we have that 

\begin{equation*}
    \theta_{k'} \rightharpoonup \bar{\theta} \text{ weakly in } W(0,T) \text{ and strongly in } L^2(Q)
\end{equation*}
We want to show that this limit is a solution to the weak form of our state system. We use the weak from as derived in Section 2.1 but now with $\theta_{k'}$ and $u_{k'}$ instead 

Take a test function $\phi \in W(0,T)$ and integrate over our space-time cylinder $Q$ we then have 
\begin{align*}
    \rho c_p \int_0^T\int_{\Omega}\frac{\partial \theta_{k'}}{\partial t}\phi \dxdt + k \int_0^T \int_{\Omega}\nabla \theta_{k'} \cdot \nabla \phi \dxdt \\ + \int_0^T \bigg (\int_{\Gamma_1}\theta_{k'}\phi \ds \bigg )u_{k'}(t) \dt =
    \int_0^T \bigg (\int_{\Gamma_1}\theta_w\phi \ds \bigg ) u_{k'} \dt 
\end{align*}
We want to see what happens if we pass to the limit $k' \rightarrow \infty$ in the equation. All terms appearing in the weak formulation are linear in $\theta_{k'}$ and $u_{k'}$. Due to the continuity of the solution $\theta_{k'}$ ($\theta_{k'} \in W(0,T) \cap C(\bar{Q})$ when we pass to the limit we get that the limit $\bar{\theta}$ solves the weak formulation of our state system with control $\bar{u}$. The solution to the state system is unique, thus we can set
\begin{align*}
    \theta(\bar{u}) = \bar{\theta} 
\end{align*}
Now by lower semicontinuity as derived in the previous lemma, we have that  
\begin{align*}
    J(\bar{\theta}, \bar{u}) \leq \lim \inf_{k'} J(\theta_{k'}, u_{k'}),
\end{align*}
and therefore the solution $(\bar{\theta}, \bar{u})$ is optimal, and hence there exists at least one optimal control.  
\end{proof}
\subsection{Real-world data for simulation}

To do our numerical simulations, we will first consider the thermal conducitivty, density and heat capacity to be constants, we divide by $\rho c_p$ in \eqref{eq:heat} to get 
\begin{equation*}
    \theta_t - \alpha \nabla \cdot (\nabla \theta) = 0 \quad \text{in $\Omega \times (0,T)$ }
\end{equation*}
Here $\alpha = \frac{k}{\rho c_p}$ is the Thermal diffusivity. To do a simulation we can set $\alpha = 3.352 mm^2/s$ to represents stainless steel, one could also simulate other materials by choosing other values of $\alpha$. To represent iron one set $\alpha_{iron} = 23 mm^2/s$. Furthermore we need the thermal conducitivity as it appears in the boundary conditions. Our choice of parameters is shown in Table \ref{tab:chosenParam}.
\begin{table}[h]
    \centering
    \caption{Parameters used for numerical simulation of the rolling of steel process.}
    \begin{tabular}{c|c}
    $\text{Parameter}$ & $\text{SI-unit}$ \\
    \hline
       $k$& $\SI{50.2}{\joule\per\metre\per\second\per\kelvin}$ \\
        $c_p$ & $\SI{460.5}{\joule\per\kilogram\per\kelvin}$ \\
        $\rho$ & $\SI{8500}{\kilogram\per\metre\cubed}$ \\
        $\theta_w$ & $\SI{20}{\celsius}$ \\
        $\theta_d$ & $\SI{700}{\celsius}$ \\
    \end{tabular}
    \label{tab:chosenParam}
\end{table}

Simulations using the parameters of Table \ref{tab:chosenParam} are shown in Figure \ref{fig:state_simulations}. These simulations were produced using a finite element method (FEM) as implemented in the Python package \verb|FEniCS| \cite{fenics}. We observe that in Figure \ref{fig:state_simulations_a} the temperature is steadily decreasing, as one would expect when continously applying a coolant to the steel. In Figure \ref{fig:state_simulations_b}, however, all of the cooling seems to happen somewhere between $t=\SI{2}{\second}$ and $t=\SI{4}{\second}$. Again, this agrees with the intuition of applying almost no coolant at the beginning or end of the cooling cycle, and maximum coolant (i.e. $u=1$) at $t=\SI{3}{\second}$. The temperatures achieved also seem to agree with values found in the literature. Thus, we conclude that the parameters we have chosen are reasonable, and that the model and the simulation of the model both perform well.
\begin{figure}
    \makebox[\linewidth]{
        \centering
    \begin{subfigure}[t]{3in}
        \includegraphics{figures/constant_u_state.pdf}
        \caption{Using the control $u(t)=1$.}
        \label{fig:state_simulations_a}
    \end{subfigure}
    ~
    \begin{subfigure}[t]{3in}
        \includegraphics{figures/gaussian_u_state.pdf}
        \caption{Here, the control is $u(t) = \exp{\left(-\left(6(t-3)/10\right)^2\right)}$; a Gaussian centered around $t=3$.}
        \label{fig:state_simulations_b}
    \end{subfigure}
    }
    \caption{Simulations using the parameters given in Table \ref{tab:chosenParam}. In addition, the initial temperature of the steel was set to \SI{1300}{\celsius}. The control $u$ used is indicated in each figure caption.}
    \label{fig:state_simulations}
\end{figure}

\subsection{Improvements on the model}

A real-world simulation for the optimal cooling process of dual phase steels (DP steels) and in particular of the type Molydbenum-Magnesium in a rolling steel process can be modelled by choosing parameter with the value 
\begin{align*}
    \theta_d = \SI{680}{\celsius}
\end{align*}
One can furthermore make the model more realistic by as in \cite{DPSteelOverview} by coupling the semilinear heat equation with an ordinary differential equation that describe the evolution of steel microstructure during the cooling process. Then one consider the optimal control problem for the controlled cooling of steel profiles in order to obtain a desired temperature and a phase distribution in the steel. Such a phase transformation can in general be described by an initial value problem of the form 
\begin{align*}
    \frac{\partial f}{\partial t} = G(f,\theta ) \\
    f_{t=0} = 0
\end{align*}
Here f is a volume fraction of the new phase, G is typically a nonlinear function of its arguments. Then one can modify the heat equation to include a right-hand side term $\rho L \frac{\partial f}{\partial t}$, where $L$ is latent heat, hence the term describes realese of heat due to the phase transformation. Furthermore one have a desired phase distribution one want to obtain, denoted $f_d(x)$ one then modify the cost functional as well to obtain an approximation to the desired phase distribution. 
\section{Numerical implementation}
To solve our optimal control problem we will consider two Black-box methods \cite{Algorithms}, and compare their performance. We will implement the simpler Gradient Descent method and compare it to Newton's method. The gradient descent method has an infamous slow convergence rate, compared to the super linear convergence rate of Newton's method. Now since both methods are Black-Box methods, we need to consider the reduced optimal control problem. Let $S$ be the solution operator for our state system. That is, let $S(u):=\theta$ so given a control $u$, $S$ transform the control into the associated temperature distribution i.e. the solution of the state system. Then we can write our cost functional on reduced form, so our objective become to minimize the reduced cost functional 
\begin{equation}
\label{eq:optimization_prob}
    \min_{u \in U}F(u) := J(S(u),u) = \frac{1}{2}\int_{\Omega}(S(u) - \theta_d)^2 \dx+ \frac{\gamma}{2}\int_0^Tu(t)^2\dt
\end{equation}


\subsection{Projected Gradient Descent}
The idea of a descent method consist of finding an optimal direction $d_k$ and stepsize $a_k$ at a given iteration $k$, where say our control is given by $u_k$. We want to find
\begin{equation*}
    F(u_k + \alpha_kd_k) < F(u_k)
\end{equation*}

One Taylor expand the LHS around $u_k$ to get a way to choose the descent direction, typically one can take different means to decide this, but a descent direction can in general be chosen by 
\begin{equation*}
    d_k = \arg\min_{||d||_U=1} \langle \nabla F(u_k), d \rangle_U
\end{equation*}
We consider gradient descent therefore we choose the descent direction by setting $d_k = -\nabla F(u_k)$. After deciding the gradient direction one need to determine how far to go in that direction, this is decided by $\alpha_k$, called \textit{line search} parameter. The ideal such parameter is 
\begin{equation*}
    a_k = \arg \min_{\alpha>0} \{ F(u_k + \alpha d_k) \}
\end{equation*}
There are different ways to choose this line search step. One require a couple of properties to be satisfied by this line search parameter these are
\begin{align*}
    F(u_k + \alpha_kd_k) < F(u_k) \text{  } \forall k \\
    F(uk + \alpha_k d_k) - F(u_k) \rightarrow 0 \text{ as } k\rightarrow \infty
\end{align*}
We stick with Armijo rule as our line search strategy. This method consists of given a descent direction $d_k$ of F at $u_k$ choose the largest $a_k \in \{ \beta^0, \beta, \beta^2,..,\beta^m,... \}$ such that
\begin{equation*}
    F (u_k + \alpha_kd_k) - F(u_k) \leq \mu \alpha_k \langle \nabla F(u_k),d_k \rangle_{U}
\end{equation*}
where $\mu,\beta  \in (0,1)$ are chosen constant, that is we use backtracking to determine the stepsize \cite{iterativeMethods}. We set $\mu = 10^{-3}$ and $\beta = \frac{1}{2}$. More facts about Armijo rule and the convergence properties of this line search strategy is in \cite{numMethods} \cite{iterativeMethods}. However we have control-constraints, so we need to ensure the argument is placed inside the admissable set of controls $U_{ad}$, in particular we have box constraints given in \eqref{eq:box_constraints}, that is we have a lower bound $u_a$ and upper bound $u_b$ for the control variable. Therefore we need to use a projection onto the set of admissable controls, that is why it is called the projected gradient descent method. We introduce a cut-off function i.e. a projection given by
\begin{equation}
    \label{eq:projection}
    \mathbb{P}_{[u_a,u_b]}(v) := \max \{u_a, \min \{u_b,v \} \}
\end{equation}
this should hold componentwise, if one allow $u_a$ and $u_b$ to be vectors, notation is similar to \cite{Algorithms}. One have to introduce this into the gradient descent method and Armijo's rule. That is we have to consider the projected Armijo's rule which is determine $\alpha_k \in \{1, \frac{1}{2},\frac{1}{4},.. \}$ such that 
\begin{equation*}
    F \bigg (\mathbb{P}_{[u_a,u_b]}(u_k + \alpha_kd_k) \bigg ) - F(u_k) \leq \mu \alpha_k \langle \nabla F(u_k),d_k \rangle_{U}
\end{equation*}
A pseudocode for the whole Projected Gradient Descent method is shown in Algorithm 1. We introduce two stopping criterion's, one based on the minimum allowed step length, and one on the difference in the computed values of the control given by 
\begin{equation}
    \label{eq:stopping}
    \tau := ||u^{(k+1)} - u^{(k)}||_U 
\end{equation}
This is to check that the controls converge towards a value, $U$ is the space of the controls. We therefore want to consider an optimization algorithm which is based on a quadratic approximation. 

\begin{codebox}
\Procname{$\proc{Algorithm1: Projected Steepest Descent with Armijo}$}
\li Choose initial value $u^{(0)}\in U_{ad}$
\li Set tolerance $\id{tol}$ and $\alpha_{min}$
\li Solve state system to obtain $\theta^{(0)}$, with $u^{(0)}$
\li Solve adjoint system to obtain $p^{(0)}$, with $u^{(0)}$ and $p^{(0)}$
\li $k=0$
\li \While ($\tau > \id{tol}$)  \Then
\li Choose the descent Direction $d_k = -\nabla F(u^{(k)})$
\li Use the Projected Armijo rule to determine $\alpha_k$
\li \If $\alpha_k < \alpha_{min}$ \Then 
\li \textbf{Break} \End
\li Set $u^{(k+1)} = \mathbb{P}_{[u_a,u_b]}\bigg (u^{(k)} + \alpha_k d_k \bigg )$
\li Solve state system to obtain $\theta^{(k+1)}$, with $u^{(k+1)}$
\li Solve adjoint system to obtain $p^{(k+1)}$, with $u^{(k+1)}$ and $\theta^{(k+1)}$
\li Compute $\tau$
\li $k = k +1$
\end{codebox}
This is the projected descent algorithm which we will apply to our control problem. The gradient of our reduced cost functional is given by

\begin{equation*}
    \nabla F(u) =  \gamma u - \frac{1}{\rho c_p} \int_{\Gamma_1} (\theta - \theta_w)p \ds = 0
\end{equation*}

The projected gradient descent method is globally convergent if the reduced cost functional F is Fréchet differentiable and bounded from below. Now since the solution to the state system $S(u) \in W(0,T) \cap C(\bar{Q})$, $F$ is indeed Fréchet differentiable. Moreover we have that $F(u) \geq 0$ for all choices of controls $u \in U_{ad}$ by the definition of $F(u)$ therefore we have global convergence. However this is a first-order descent method, and the convergence rate is typically very slow, since it solely rely on first-order gradient information. We expect to see pretty fast convergence when the control approximated thus far $u_k$ is far from the optimal, and the convergence gets slower and slower once we get closer to the optimal $\bar{u}$ 

\subsection{Newton's method}
Another type of direction is obtained at each direction $d_k$ if one instead use a quadratic expansion. For Newton's method we need again the reduced cost function $F$ to be Fréchet differentiable and that its Fréchet derivative is boundedly invertible $\forall u \in U_{ad}$, or at least close to the true solution $\bar{u}$. We try to approximate a solution of the equation $F(u) = 0$ by repeatedly linearizing the function and solving the linearized system
\begin{align}
    \label{eq:Newton}
    u_{k+1} = u_k - F'(u_k)^{-1}F(u_k)
\end{align}
Want to use Newton's method with a primal-dual active set strategy to handle the control constraint. In the unconstrained control case for an optimal control problem, a necessary optimality condition for the reduced control problem is that $\nabla F(u)$ vanishes. However in the presence of control constraints, this condition must be replaced by the variational inequality
\begin{equation}
    \langle \nabla F(\bar{u}), (v-\bar{u}) \rangle \geq 0 \text{ } \forall v \in U_{ad}
\end{equation}
We can reformulate this using Lagrange multipliers, requiring 
\begin{equation*}
    \nabla F(u) + \mu_b - \mu_a =0 
\end{equation*}
to get the complementary slackness conditions which are given by the following equations 
\begin{align}
    \label{eq:complementary}
    \mu_a \geq 0, \quad u_a - u \leq 0, \quad \mu_a(u_a - u) = 0 \\
    \mu_b \geq 0, \quad u - u_b \leq 0, \quad \mu_b(u-u_b) =0
\end{align}
To treat \eqref{eq:complementary} numerically, one combine this into one equation, by introducing a new Lagrange multiplier $\mu = \mu_b - \mu_a$, one convert the complementary slackness conditions into one equation involving maximum and minimum. Therefore the system become 
\begin{align}
    \label{eq:finalComp}
    \nabla F(u) + \mu = 0 \\
    \max \{0, \mu + c(u-u_b) \} + \min \{0, \mu + c(u-u_b) \} - \mu = 0
\end{align}

The latter equation is just a reformulation of the two equations in \eqref{eq:complementary}. Newton's method is based on calculating the Hessian of the reduced cost functional, since it uses second order information about the functional. This depends on the primal variable $u$ and the dual variable $\mu$ we introduce the active and inactive sets at an iterate $(u_k, \mu_k)$ by 
\begin{align}
    \label{eq:active_inactive}
    A_k^{+} := \{ t \in [0,T]: \text{ } \mu_k + c(u_k - u_b) >0 \} \\
    A_k^{-} := \{ t \in [0,T]: \text{ } \mu_k + c(u_k - u_a) <0 \} \\
    I_k := [0,T] \backslash A_k \quad A_k := A_k^{+} \cup A_k^{-}
\end{align}

The Newton direction at an iterate $(u_k, \mu_k)$ can be calculated by solving the following symmetric form of the Newton system

\begin{equation}
    \label{eq:Newton_system}
    \begin{bmatrix}
        \nabla^2 F(u_k) & \chi_{A_k} \\
        \chi_{A_k} & 0 
    \end{bmatrix}
    \begin{bmatrix}
    \delta u \\
    \delta \mu|_{A_k}
    \end{bmatrix}
    = - \begin{bmatrix}
    \nabla F(u_k) + \mu_k \\
    \chi_{A_k^{+}}(u_k - u_b) + \chi_{A_k^{-}}(u_k - u_a)
    \end{bmatrix}
\end{equation}
The next dual variable $\mu_{k+1}$ is set to be zero on the the inactive set $I_k$. A Pseudocode for Newton's method with primal-dual active set strategy is given in Algorithm 2

\begin{codebox}
\Procname{$\proc{Algorithm2: Newton's method with Primal-Dual Active Set Strategy}$}
\li Choose initial value $u^{(0)} \in U_{ad}$ and $\mu_0$ 
\li Set tolerance $tol$
\li Set $k = 0$
\li Solve state system to obtain $\theta^{(0)}$
\li Solve adjoint system to obtain $p^{(0)}$
\li \While $(\tau > tol)$ \Then 
\li Evaluate the reduced gradient $\nabla F(u_k)$
\li Compute the active sets $A_k^{+}$ and $A_k^{-}$
\li \If $A_k^{+} = A_{k-1}^{+}$ and $A_{k-1}^{-} = A_k^{-}$ \Then 
\li \textbf{Break} \End
\li Solve the Newtonian system given in \eqref{eq:Newton_system} iteratively
\li Set $u^{(k+1)} = u^{(k)} + \delta u$  
\li Set 
\begin{equation*}
    \mu_{k+1}(t) = 
    \begin{cases}
     u_k + \delta u \text{ if } t \in A_k \\
     0 \qquad \text{ if } t \in I_k
     \end{cases}
\end{equation*}
\li Solve state system to obtain $\theta(^{(k+1)}$ using $u^{(k+1)}$
\li Solve adjoint system to obtain $p^{(k+1)}$ 
\li Compute $\tau$
\li Set $k = k+1$
\end{codebox}

Where $\tau$ is calculated as in Algorithm 1. We do not globalise the algorithm here, but that can be done to achieve convergence from an arbitrary starting point. This is done by for instance carrying out a few gradient descent steps in the starting phase, and then switching over to Newton's method. The notion used here is similar to that in \cite{Algorithms}. 

Expect to observe superior convergence properties of Newton's method compared to Steepest Descent, as Newton's method is superlinearly convergent compared to $q$-linearly convergent. The rate of convergence can be measured similarly as in \cite{DPSteel} by setting 
\begin{equation}
    \label{eq:rate_of_conv}
    e_k := \frac{||u^{(k)}-\bar{u}||_U + ||\theta^{(k)}-\bar{\theta}||_{L^2(Q)} +||p^{(k)}-\bar{p}||_{L^2(Q)} }{||u^{(k+1)}-\bar{u}||_U + ||\theta^{(k+1)}-\bar{\theta}||_{L^2(Q)} +||p^{(k+1)}-\bar{p}||_{L^2(Q)}}
\end{equation}
where $(\bar{\theta},\bar{u},\bar{p})$ is the analytical solution to the control problem. We use this as a convergence test of the two algorithm on a control problem where we can find an analytical solution. 

Other mentionable methods to use on the optimal control problem is the sequential quadratic programming (SQP) method \cite{Algorithms}, and reduced SQP \cite{DPSteel}. 

\section{Test cases}

In the case of dual phase steel we can consider a more realistic scenario used in the industry. To do so we introduce two desired temperatures $\theta_{d_1}$ and $\theta_{d_2}$. For the optimal cooling of dual phase steel there is a phase transition in the higher temperature regime, this is modelled by the desired temperature $\theta_{d_1}$ which we want to be attained in a subset $[c_1T, c_2T]$ of $[0,T]$ with $c_2 <1$. Then we expect the optimal cooling to lead to a rapid cooling towards $\theta_{d_1}$ and stay almost constant until $t=c_2T$. Then we also expect the optimal cooling strategy to lead to a rapid cooling towards the new final desired temperature $\theta_{d_2}$. In order to model this scenario we need to introduce an extra term to the cost functional. Consequently we will get a new adjoint system, however the expression for the gradient stays the same, as the new term does not involve the control, so we can apply the Projected gradient method in Algorithm 1 almost without doing any modifications. The new cost functional is given by 
\begin{equation}
\begin{aligned}
    \label{eq:new_cost}
    J(\theta,u) : =& \frac{\alpha_1}{2}\iint_{Q}\chi_{[c_1T,c_2T]}(\theta(x,t) - \theta_{d_1}(x,t))^2 \dxdt \\
    &+ \frac{\alpha_2}{2}\int_{\Omega} (\theta(x,T) - \theta_{d_2})^2 \mathop{dx} + \frac{\gamma}{2}\int_0^T|u(t)|^2\dt
\end{aligned}
\end{equation}
Here $\chi_E$ denotes the characteristic function for the set $E$. For convenience we formulate the whole optimal control problem in this case. 
\begin{equation*}
\begin{aligned}
   \min J(\theta, u) =& \frac{\alpha_1}{2}\iint_{Q}\chi_{[c_1T,c_2T]}(\theta(x,t) - \theta_{d_1}(x,t))^2 \dxdt \\
   &+ \frac{\alpha_2}{2} \int_\Omega (\theta(x, T) - \theta_{d_2}(x))^2 \mathop{dx} + \frac{\gamma}{2} \int_0^{T} |u(t)|^2 \mathop{dt} 
\end{aligned}
\end{equation*}
subject to
\begin{align*}
       \rho c_p \theta_t - \nabla \cdot (k \nabla \theta) &= 0 \quad &\text{in } Q  \\
      -k \frac{\partial \theta}{\partial \nu} &= u(t) (\theta - \theta_w) \quad &\text{on } \Sigma_1, \\
      -k \frac{\partial \theta}{\partial \nu} &= 0 \quad &\text{on } \Sigma_0, \\
      \theta(x, 0) &= \theta_0 &
\end{align*}

We again use the formal Lagrange method to derive the adjoint system. The lagrangian takes the same form except an additional term which gives an extra contribution to the adjoint system. Taking the Frechet derivative with respect to the state of the Lagrangian we get

\begin{equation}
  \begin{aligned}
  0 = \L_\theta(\theta, u, p)h = \alpha_2 \int_\Omega (\theta(x,T) - \theta_{d_2})h(x, T)\dx - \iint_Q h_t p\dxdt \\
  - \frac{k}{\rho c_p}\iint_Q\nabla h\nabla p \dxdt
  - \frac{1}{\rho c_p}\iint_{\Sigma_1} u(t)h p\dsdt \\
  + \alpha_1 \iint_{Q}\chi_{[c_1T,c_2T]}(\theta - \theta_{d_1})h \dxdt
  \end{aligned}
\end{equation}

Then we may use partial integration, and a similar argument as used to derive the adjoint system \eqref{eq:adjoint-system} to get the adjoint system in this case. The adjoint system is given by 

\begin{subequations}
   \begin{align*} 
      \rho c_p p_t + k\Delta p + \chi_{[c_1T,c_2T]}\alpha_1(\theta - \theta_{d_1})&= 0 \quad\qquad\textrm{ in } Q  \\
      {-k}\frac{\partial p}{\partial\nu} &= u(t)p \,\,\quad\textrm{ on } \Sigma_1  \\
      {-k}\frac{\partial p}{\partial\nu} &= 0 \,\quad\qquad\textrm{ on } \Sigma_0  \\
      \rho c_p p(x, T) &= \alpha_2(\theta(x, T) - \theta_{d_2}). \quad \textrm{ in } \Omega
   \end{align*}
\end{subequations}

Now we see that we got the additional term $\chi_{[c_1T,c_2T]}\alpha_1(\theta-\theta_{d_1})$ in the PDE, thus the adjoint state $p$ depends on the temperature distribution on the interval $[c_1T,c_2T]$. As a consequence we expect to observe a non-constant control u, as mentioned we expect a rapid decay towards $\theta_{d_1}$ then a constant $u$ until $t=c_2T$ and then again a rapid decay towards $\theta_{d_2}$. Since we only consider cooling by water, we need to require $\theta_{d_1} \geq \theta_{d_2}$.



\subsection{Further improvements to the model}
One can make our model more realistic as in \cite{DPSteelOverview} by coupling the semilinear heat equation with an ordinary differential equation that describe the evolution of steel microstructure during the cooling process. Then one consider the optimal control problem for the controlled cooling of steel profiles in order to obtain a desired temperature and a phase distribution in the steel, so one get an additional variable to control. Such a phase transformation in the steel microstructure can in general be described by an initial value problem of the form 
\begin{align*}
    \frac{\partial f}{\partial t} = G(f,\theta ) \\
    f_{t=0} = 0
\end{align*}
Here f is a volume fraction of the new phase, G is typically a nonlinear function of its arguments, so the phase-developement depends on temperature and the volume fraction of the phase already present. One start with zero of the given phase. 

Then one can modify the heat equation to include a right-hand side term $\rho L \frac{\partial f}{\partial t}$, where $L$ is latent heat, that is a quantity describing the release of heat due to the phase transformation. One can also vary the coolant profile, that is applied different amount of water to the different parts of the steel slab, this can be given by a function $\beta (x)$. The parabolic PDE for the temperature evolution then become 
\begin{align*}
    \rho c_p \theta_t - \nabla \cdot (k\nabla \theta) = \rho L f_t \quad \textrm{ in } Q \\
    - k \frac{\partial \theta}{\partial n} = u(t)\beta(x) \bigg (\theta - \theta_w \bigg ) \quad \textrm{ on } \Sigma_1 \\
    -k \frac{\partial \theta}{\partial n} = 0 \quad \textrm{ on } \Sigma_2 \\
    \theta(x,0) = \theta_0(x) \quad \textrm{ in } \Omega
\end{align*}

Furthermore one have a desired phase distribution one want to obtain, denoted $f_d(x)$ one then modify the cost functional as well to obtain an approximation to the desired phase distribution. That is one get a cost functional of the form 
\begin{equation*}
    J(\theta, f, u) := \frac{\alpha_1}{2}\int_{\Omega}(f(x,T)-f_d(x))^2\dx + \frac{\alpha_2}{2}\int_{\Omega}(\theta(x,T) - \theta_d(x))^2\dx + \frac{\alpha_3}{2}\int_0^Tu(t)^2 \dt
\end{equation*}

Then choosing the appropriate function $G(f,\theta)$ depends on what phase-transformation one consider, and $L$ also depends upon the chosen phase-transformation. The coolant profile $\beta(x)$ must also be chosen. Furthermore one need to choose the desired temperature profile $\theta_d(x)$ and the desired volume-fraction profile over the microstructure $f_d(x)$. All these depending on the particular application, making this a more general model than what we have been considering. 


\printbibliography
% \bibliographystyle{alphabetic}
% \bibliography{ref}  

\end{document}
