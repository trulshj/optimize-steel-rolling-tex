\subsection{Real-world data for simulation}

To do our numerical simulations, we will consider the thermal conducitivty, density and heat capacity to be constants, we divide by $\rho c_p$ in \eqref{eq:heat} to get 
\begin{equation*}
    \theta_t - \alpha \nabla \cdot (\nabla \theta) = 0 \quad \text{in $\Omega \times (0,T)$ }
\end{equation*}
Here $\alpha = \frac{k}{\rho c_p}$ is the Thermal diffusivity. Now this is a simplifiction of the model, as more realistically the heat conductivity is temperature dependent i.e. $k = k(\theta)$ and the heat capacity depends on heat conductivity $c_p = c_p(k)$. We set the parameters as shown in Table \ref{tab:chosenParam} when doing our simulation of the heat evolution in our multiphase steel, values are similar to \cite{DPSteel}. 


\begin{table}[h]
    \centering
    \caption{Parameters used for numerical simulation of the rolling of steel process.}
    \begin{tabular}{c|c}
    $\text{Parameter}$ & $\text{SI-unit}$ \\
    \hline
       $k$& $\SI{50.2}{\joule\per\metre\per\second\per\kelvin}$ \\
        $c_p$ & $\SI{509.6}{\joule\per\kilogram\per\kelvin}$ \\
        $\rho$ & $\SI{7850}{\kilogram\per\metre\cubed}$ \\
        $\theta_w$ & $\SI{20}{\celsius}$ \\
        $\theta_d$ & $\SI{700}{\celsius}$ \\
        $\theta_0$ & $\SI{1200}{\celsius}$ \\
    \end{tabular}
    \label{tab:chosenParam}
\end{table}
Here $\theta_0$ represent the initial temperature of the steel slab shown in \ref{fig:steel_slab}, while $\theta_w$ is the temperature of the applied coolant. To have a realistic size of the slab one might choose $\Omega = (0,7.5)\times(0,0.7) \text{cm}^2$

To test our solution procedure of the state system, we use the parameters of Table \ref{tab:chosenParam} togheter with two dummy choices of the control $u(t)$. These are shown in Figure \ref{fig:state_simulations}. These simulations were produced using a finite element method (FEM) as implemented in the Python package \verb|FEniCS| \cite{fenics}. For the first choice we sat $u(t) = 1$ we observe that in Figure \ref{fig:state_simulations_a} the temperature is steadily decreasing, as one would expect when continously applying a coolant to the steel. In Figure \ref{fig:state_simulations_b}, where we have $u(t) =\exp{\left(-\left(6(t-3)/10\right)^2\right)}$, all of the cooling seems to happen somewhere between $t=\SI{2}{\second}$ and $t=\SI{4}{\second}$, which agrees with a gaussian distribution as nearly all mass is centered in this region. Again, this agrees with the intuition of applying almost no coolant at the beginning or end of the cooling cycle, and maximum coolant (i.e. $u=1$) at $t=\SI{3}{\second}$. The temperatures achieved also seem to agree with values found in the literature. Thus, we conclude that the parameters we have chosen are reasonable, and that the model and the simulation of the state system both perform well.
\begin{figure}
    \makebox[\linewidth]{
        \centering
    \begin{subfigure}[t]{3in}
        \includegraphics{figures/constant_u_state.pdf}
        \caption{Using the control $u(t)=1$.}
        \label{fig:state_simulations_a}
    \end{subfigure}
    ~
    \begin{subfigure}[t]{3in}
        \includegraphics{figures/gaussian_u_state.pdf}
        \caption{Here, the control is $u(t) = \exp{\left(-\left(6(t-3)/10\right)^2\right)}$; a Gaussian centered around $t=3$.}
        \label{fig:state_simulations_b}
    \end{subfigure}
    }
    \caption{Simulations using the parameters given in Table \ref{tab:chosenParam}. In addition, the initial temperature of the steel was set to \SI{1300}{\celsius}. The control $u$ used is indicated in each figure caption.}
    \label{fig:state_simulations}
\end{figure}
