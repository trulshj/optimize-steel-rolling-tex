\section{Simulation data and test case}

In our numerical simulations we use the simplification that the thermal conductivity, density and heat capacity of the steel are constants. This is a simplification of reality, as the heat conductivity is temperature dependent, i.e. $k = k(\theta)$, and the heat capacity will depend on the heat conductivity, $c_p = c_p(k)$.

% Trenger dette å være med?
% Dividing \eqref{eq:heat-in-omega} by $\rho c_p$, we get 
% \begin{equation*}
%     \theta_t - \alpha \nabla \cdot (\nabla \theta) = 0 \quad \text{in $\Omega \times (0,T)$. }
% \end{equation*}
% Here, $\alpha \coloneqq \frac{k}{\rho c_p}$ is the thermal diffusivity.

We set the parameters as shown in Table \ref{tab:chosenParam} when doing our simulation of the heat evolution in our multiphase steel, values are similar to \cite{DPSteel}. 
\begin{table}[h]
    \centering
    \caption{Parameters used for numerical simulation of the rolling of steel process.}
    \begin{tabular}{@{}lr@{}} \toprule
    $\text{Parameter}$ & $\text{SI-unit}$ \\
    \midrule
       $k$& $\SI{50.2}{\joule\per\metre\per\second\per\kelvin}$ \\
        $c_p$ & $\SI{509.6}{\joule\per\kilogram\per\kelvin}$ \\
        $\rho$ & $\SI{7850}{\kilogram\per\metre\cubed}$ \\
        $\theta_w$ & $\SI{20}{\celsius}$ \\
        $\theta_d$ & $\SI{700}{\celsius}$ \\
        $\theta_0$ & $\SI{1300}{\celsius}$ \\ \bottomrule
    \end{tabular}
    \label{tab:chosenParam}
\end{table}
Here $\theta_0$ represent the initial temperature of the steel slab shown in \ref{fig:steel_slab}, while $\theta_w$ is the temperature of the applied coolant. To have a realistic size of the slab one might choose $\Omega = (0,7.5)\times(0,0.7) \text{cm}^2$.

To test our simulation of the state system, we used the parameters of Table \ref{tab:chosenParam} with various choices of the control $u$. We then verified that the solutions behave as expected. One such example is shown in Figure \ref{fig:state_simulations}.

The simulations were produced using a finite element method (FEM) as implemented in the Python package \verb|FEniCS|~\cite{fenics}. With $u(t) = 1$  as in Figure \ref{fig:state_simulations_a}, we observe that  the temperature is steadily decreasing, as one would expect when continuously applying coolant to the steel. Other runs with different choices of $u$ also behaved as one would expect.

Thus, we are confident that the parameters we have chosen are reasonable, and that the model and the numerical implementation of it both perform well.
\begin{figure}
    \makebox[\linewidth]{
        \centering
    \begin{subfigure}[t]{3.5in}
        \includegraphics{figures/uniform_u_state.pdf}
        \caption{The state $\theta$ as solved at different time points.}
        \label{fig:state_simulations_a}
    \end{subfigure}
    ~
    \begin{subfigure}[t]{3.5in}
        \includegraphics{figures/uniform_u_adjoint.pdf}
        \caption{The adjoint $p$ as solved at different time points.}
        \label{fig:state_simulations_b}
    \end{subfigure}
    }
    \caption{Simulations using the parameters given in Table \ref{tab:chosenParam}. A constant control $u(t) = 1$ was used.}
    \label{fig:state_simulations}
\end{figure}
