


\subsection{Further improvements to the model}
One can make our model more realistic as in \cite{DPSteelOverview} by coupling the semilinear heat equation with an ordinary differential equation that describe the evolution of steel microstructure during the cooling process. Then one consider the optimal control problem for the controlled cooling of steel profiles in order to obtain a desired temperature and a phase distribution in the steel, so one get an additional variable to control. Such a phase transformation in the steel microstructure can in general be described by an initial value problem of the form 
\begin{align*}
    \frac{\partial f}{\partial t} = G(f,\theta ) \\
    f_{t=0} = 0
\end{align*}
Here f is a volume fraction of the new phase, G is typically a nonlinear function of its arguments, so the phase-developement depends on temperature and the volume fraction of the phase already present. One start with zero of the given phase. 

Then one can modify the heat equation to include a right-hand side term $\rho L \frac{\partial f}{\partial t}$, where $L$ is latent heat, that is a quantity describing the release of heat due to the phase transformation. One can also vary the coolant profile, that is applied different amount of water to the different parts of the steel slab, this can be given by a function $\beta (x)$. The parabolic PDE for the temperature evolution then become 
\begin{align*}
    \rho c_p \theta_t - \nabla \cdot (k\nabla \theta) = \rho L f_t \quad \textrm{ in } Q \\
    - k \frac{\partial \theta}{\partial n} = u(t)\beta(x) \bigg (\theta - \theta_w \bigg ) \quad \textrm{ on } \Sigma_1 \\
    -k \frac{\partial \theta}{\partial n} = 0 \quad \textrm{ on } \Sigma_2 \\
    \theta(x,0) = \theta_0(x) \quad \textrm{ in } \Omega
\end{align*}

Furthermore one have a desired phase distribution one want to obtain, denoted $f_d(x)$ one then modify the cost functional as well to obtain an approximation to the desired phase distribution. That is one get a cost functional of the form 
\begin{equation*}
    J(\theta, f, u) := \frac{\alpha_1}{2}\int_{\Omega}(f(x,T)-f_d(x))^2\dx + \frac{\alpha_2}{2}\int_{\Omega}(\theta(x,T) - \theta_d(x))^2\dx + \frac{\alpha_3}{2}\int_0^Tu(t)^2 \dt
\end{equation*}

Then choosing the appropriate function $G(f,\theta)$ depends on what phase-transformation one consider, and $L$ also depends upon the chosen phase-transformation. The coolant profile $\beta(x)$ must also be chosen. Furthermore one need to choose the desired temperature profile $\theta_d(x)$ and the desired volume-fraction profile over the microstructure $f_d(x)$. All these depending on the particular application, making this a more general model than what we have been considering. 