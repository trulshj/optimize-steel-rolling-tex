\section{Existence and uniquenes for the optimal control problem}\label{proof}

\subsection{Existence and uniqueness state system}
Need to consider the existence of a solution to the state equation. In order to do so we consider a more general linear parabolic partial differential equation, and derive existence for that PDE, and use that to conclude about the existence for our particular problem as well. Consider a general parabolic PDE with mixed boundary conditions

\begin{align}
    \label{eq:moreGen}
    u_t - Au + c_0 y = f \quad \text{in } Q = \Omega \times (0,T) \\
    \partial_{n_A} u + \alpha u = g \quad \text{on } \Sigma = \Gamma \times (0,T) \\
    u(.,0) = u_0(.) \quad \text{in } \Omega
\end{align}
It does not predescribe any difficulties by splitting the boundary $\Gamma$ into disjoint parts, and predescribe Neumann conditions on them separately. The functions $\alpha$, $\beta$, $f$ and $g$ all depend on $(x,t)$. Here $A$ is an elliptic differential operator of the following form
%
\begin{equation*}
    Af(x) := -\sum_{i,j}^N\frac{\partial}{\partial x_i}(a_{ij}(x)\frac{\partial}{\partial x_j}f(x)) \quad x\in \Omega
\end{equation*}
One need some requirements on the coefficents of the matrix $A=(a_{ij})$, that is one require $a_{ij} \in L^{\infty}(\Omega)$ and that they are symmetric i.e. $a_{ji}(x) = a_{ij}(x)$ for $x\in \Omega$. Moreover one also assumes that the uniform ellipticity condition is satisfied, that is $\exists M>0$ such that 
\begin{equation*}
    \label{eq:uniformEl}
    \sum_{i,j}^N a_{ij}(x)x_i x_j \geq M|x|^2 \quad \forall x \in \mathbb{R}^N \text{ for a.e. $x\in \Omega$}
\end{equation*}
Moreover from \eqref{eq:moreGen} we see that we have $\partial_{n_A}$ appearing which is the directional derivative of the conormal vector $n_{A}$ which is a vector whose component are given by $n_{A} = An$ where A is the elliptic differential operator. Then we can infer an existence and uniqueness theorem for the PDE given in \eqref{eq:moreGen}. 

\begin{theorem}[Existence and uniqueness] Assume that $\Omega \subset \mathbb{R}^n$ is a bounded Lipschitz domain with boundary $\Gamma$ and let $T>0$ denote the final time. Moreover if $c_0 \in L^{\infty}(Q)$ and $\alpha \in L^{\infty}(\Sigma)$, where $\alpha(x,t) \geq 0$ for a.e. $(x,t) \in \Sigma$, $y_0 \in L^2(\Sigma), f \in L^2(Q)$ and $g \in L^2(\Sigma)$. Then the parabolic initial-value problem \eqref{eq:moreGen} has a unique weak solution in $W_2^{1,0}(Q)\cap L^{\infty}(0,T;L^2(\Omega))$. Moreover, there is a constant C>0 which is independent of $f$, $g$ and $u_0$ such that 
\begin{equation*}
    \max_{t \in [0,T]}\|u(.,t)\|_{L^2(\Omega)} + \|u\|_{W_2^{1,0}(Q)} \leq C(\|f\|_{L^2(Q)} + \|g\|_{L^2(\Sigma)} + \|u_0\|_{L^2(\Omega)})
\end{equation*}
this holds for all $f \in L^2(Q), g \in L^2(\Sigma)$ and $u_0 \in L^2(\Omega)$
\end{theorem}

\begin{proof}
The proof is given by Tröllsch in \cite{optimalControl} under theorem 7.8. However we can give a rough sketch of the required steps in the proof. WOLOG one may assume that $c_0(x,t)\geq 0$ for a.e. $(x,t) \in Q$, since if that were not the case one could do the substitution $y(x,t) \rightarrow e^{\lambda t}\bar{y}(x,t)$. Then the resulting differential equation for $\bar{y}$ involves the term $(\lambda + c_0)\bar{y}$ instead of $c_0y$ as in the original PDE. Taking $\lambda >0$ large enough this will certainly be positive. 

Now in order to prove the existence in the theorem one need to proceed in 4 steps,
\begin{enumerate}
    \item Do a Galerkin approximation to the problem
    \item Estimate the sequence $\{y_n \}$ which is an approximation sequence for the states of the PDE
    \item Consider the convergence of the sequence of estimated controls $\{u_j^N\}_j$ and states $\{ y_n \}$
    \item Show that the limit $y_n \rightarrow y$ is indeed a weak solution to the PDE
\end{enumerate}
One use densness
To prove uniqueness one have to use an energy inequality together with the regularity of the solution.
\end{proof}


From this theorem we can conclude about existence of a solution to our particular parabolic PDE. Set $c_0 = 0$, $A = \frac{1}{\rho c_p}\nabla \cdot (k\nabla)$ , $f = 0$, then $\partial_{n_A} = \partial_n$ furthermore we set $\Gamma = \Gamma_1 \cup \Gamma_2$, to partition our boundary into to disjoint separate parts, now let $\alpha_1 = -\frac{u(t)}{k}$ and $g_1 = -\frac{u(t)\theta_w}{k}$, while one set $\alpha_2 = 0, g_2 = 0$, where $\alpha_i$ is the value of $\alpha$ on $\Sigma_i$ and similarly for $g_i$. This results in the next corollary. 

\begin{corollary}[Existence]
Suppose that $\theta_w \in L^{\infty}(\Sigma_1)$, $\theta_0 \in L^2(\bar{\Omega})$, $u \in L^{\infty}(0,T)$ and $u\geq 0$ then the initial value problem defined in \eqref{eq:heat} admits a unique solution $\theta \in H^{1,0}(Q)$ after a possible modification on a nullset, we have $\theta \in W(0,T)$. Morover this weak solution to the PDE satisfy and upper bound of the form 
\begin{equation*}
    \|\theta \|_{W(0,T)} \leq C\bigg ( \|g_1\|_{L^2(\Sigma)} + \|\theta_0\|_{L^2(\Sigma)} \bigg )
\end{equation*}
for a constant $C>0$ independent on $(g_1, \theta_0)$
\end{corollary}

Actually using theorem 5.5 in \cite{optimalControl} we get that our solution 
\begin{equation*}
    \theta \in W(0,T) \cap C (\bar{Q})
\end{equation*}
and as we will see the solution to the adjoint system satisfy have the same regularity using a time transformation. 


\subsection{Existence and uniqueness Adjoint system}
Want to prove that there exist a unique weak solution $p \in W_2^{1,0}(Q)$ to our adjoint system, and modifying $p$ on a nullset we get $p\in W(0,t)$. We consider the sightly more general parabolic initial-boundary value problem 
\begin{align*}
    -p_t -\nabla^2p +c_0p = a_Q \qquad \text{in } Q \\
    \partial_np + \alpha p = a_{\Sigma} \qquad \text{on } \Sigma \\
    p|_{t=T} = a_{\Omega} \qquad \text{in } \Omega
\end{align*}
if we assume bounded and measurable coefficent functions i.e. $c_0$ and $\alpha$, furtherore $a_Q \in L^2(Q)$, $a_{\Sigma} \in L^2(\Sigma)$ and $a_{\Omega} \in L^2(\Omega)$ then if we multiply by a test function and integrate, we can introduce a bilinear form
\begin{equation}
    \label{eq:parabolic_adj}
    A(y,v)(t) := \int_{\Omega}(\nabla y \cdot \nabla v + c_0(\cdot,t)yv) \dx + \int_{\Gamma}\alpha(\cdot,t)yv \ds
\end{equation}
Then we have a well-posedness result for this PDE

\begin{lemma}[Well-posedness]
The parabolic adjoint system given in \eqref{eq:parabolic_adj} has a unique weak solution $p \in W_2^{1,0}(Q)$ which is a solution to the variational problem 
\begin{equation*}
    \iint_Q pv_t \dxdt + \int_0^TA(p,v)(t)\dt = \int_{\Omega}a_{\Omega}v(T) \dt + \iint_Q a_Qv \dxdt + \iint_{\Sigma}a_{\Sigma}v \dsdt
\end{equation*}
this should hold $\forall v \in W_2^{1,1}(Q)$ with $v|_{t=0} = 0$. Now if we modify p on a nullset we have $p\in W(0,T)$ and $\exists$ a constant $M>0$ that does not depend on $(a_Q,a_{\Sigma}, a_{\Omega})$ such that 
\begin{equation*}
    \|p\|_{W(0,T)} \leq M \bigg (\|a_Q\|_{L^2(Q)} + \|a_{\Sigma}\|_{L^2(\Sigma)} + \|a_{\Omega}\|_{L^2(\Omega)} \bigg ).
\end{equation*}
\end{lemma}

\begin{proof}
The idea is to reduce it to a forward parabolic initial-boundary value problem, so we do a time transformation, take $\tau \in [0,T]$ and introduce 
\begin{equation*}
    \hat{p}(\tau) := p(T-\tau)
\end{equation*}
one does similarly with $\hat{v}(\tau)$. Then one have $\hat{p}(0) = p(T)$ and $\hat{p}(T) = p(0)$ similarly for $\hat{v}$. We have that $\hat{a_Q}(.,t):= a_Q(.,T-\tau)$ likewise for all the other coefficents, due to this time transform we have that
\begin{equation*}
    \iint_Qpv_t \dxdt = - \iint_Q \hat{p}\hat{v_{\tau}} \dxdt
\end{equation*}
Considering the weak formulation, it now corresponds to the forward parabolic initial-boundary value problem given by 
\begin{align*}
    \hat{p_{\tau}} - \nabla^2 \hat{p} + c_0 \hat{p} = \hat{a_Q} \qquad \text{in } Q \\
    \partial_n \hat{p} + \alpha \hat{p} = \hat{a_{\Sigma}} \qquad \text{on } \Sigma \\
    \hat{p}(0) = \hat{a_{\Omega}} \qquad \text{in } \Omega
\end{align*}
Now by the previous existence and uniqueness theorem this problem admits a unique weak solution $\hat{p}$ belonging to $W(0,T)$. Reversing the time transformation we have proven uniqueness and existence for the adjoint system
\end{proof}
Now this system is more general than ours, so we can reduce it to our case. Consider \eqref{eq:adjoint-eqn}, if we set $c_0 = 0$, $a_Q = 0$ and split up the boundary $\Sigma = \Sigma_1 \cup \Sigma_2$ where we set $\alpha_1 = \frac{u(t)}{k}$ and $\alpha_2 = 0$, $a_{\Sigma_i}=0$ for $i\in \{1,2 \}$ and finally set $a_{\Omega} = \theta|_{t=T}-\theta_d$ we have reduced to the case of the lemma, consequently we have existence and uniqueness for \eqref{eq:adjoint-eqn}.



\subsection{Existence of optimal control}
Under the required conditions for a solution to \eqref{eq:heat} we can go on to prove the existence of an optimal control $\bar{u}$. 

\begin{lemma}[WLSC]
    The cost functional given b y
    \begin{equation*}
         J(\theta, u) = \frac{1}{2} \int_\Omega (\theta(x, T) - \theta_d)^2 \mathop{dx} + \frac{\gamma}{2} \int_0^{T} u^2 \mathop{dt}
    \end{equation*}
    is weakly lower semicontinous (WLSC). 
\end{lemma}

\begin{proof}
We can reformulate the cost functional $J(\theta, u)$ in terms of norms as 
\begin{equation*}
    J(\theta, u) = \frac{1}{2}\|\theta (,T) - \theta_d \|_{L^2(\Omega)} + \frac{\gamma}{2}\|u\|_{L^2(0,T)}
\end{equation*}
Now a norm is convex by applying the triangle inequality and continuity of the norm follows by an application of the reverse triangle inequality. This thus implies that the cost functional is weakly lower semicontinous. 
\end{proof}

\begin{theorem}[Optimal Control]
If the assumptions in the previous theorem are satisfied, then there exist at least one solution to the optimal control problem 
\begin{align*}
       \min J(\theta, u) = \frac{1}{2} \int_\Omega (\theta(x, T) - \theta_d)^2 \mathop{dx} \mathop{dt} + \frac{\gamma}{2} \int_0^{T} u^2 \mathop{dt} \\
       \text{subject to} \\
       \rho c_p \theta_t - \nabla \cdot (k \nabla \theta) &= 0 \quad &\text{in } Q  \\
      -k \frac{\partial \theta}{\partial \nu} &= u(t) (\theta - \theta_w) \quad &\text{on } \Sigma_1, \\
      -k \frac{\partial \theta}{\partial \nu} &= 0 \quad &\text{on } \Sigma_0, \\
      \theta(x, 0) &= \theta_0 &
\end{align*}
\end{theorem}

\begin{proof}
We follow the same footsteps as in  \cite{DPSteel}. By the existence corollary for the parabolic PDE we know that  for any $u \in U_{ad} \text{ } \exists \text{ }\theta \in W(0,T)\cap C(\bar{Q})$. The set of admissible controls $U_{ad}$ is bounded in $L^{\infty}(0,T)$ therefore the solution $\theta$ is bounded in $W(0,T) \cap C(\bar{Q})$ thus the cost functional will be bounded. We can as a consequence assume we have a minimising sequence which we represent by the tuple $\{(\theta_k,u_k)\}_{k\in \mathbb{N}}$ such that we have 
\begin{equation*}
    \lim_{k\rightarrow \infty}J(\theta_k,u_k) = \inf_{(\theta,u)\in (W(0,T) \cap C(\bar{Q})\times U_{ad}}J(\theta,u)
\end{equation*}
where $(\theta_k,y_k) = S(u_k)$ where $S(.)$ is the solution operator which assigns the solution to the state system for a given control $u_k$.

Now $U_{ad}$ is a closed linear subspace of the Banach space $L^{\infty}(0,T)$, and is therefore itself a  Banach space, furthermore $U_{ad}$ is bounded and convex, so the sequence $\{u_k\}$ will be bounded, and there exist a weakly convergent subsequence of the controls $\{u_{k'} \} \subset \{u_k \}$ and a limit $\bar{u} \in L^{\infty}(0,T)$ such that
\begin{equation*}
    u_{k'} \rightharpoonup \bar{u} \text{ in } L^2(0,T)
\end{equation*}

Now since $U_{ad}$ is closed and convex, $U_{ad}$ is weakly closed, therefore $\bar{u} \in U_{ad}$. \bigskip

Using our existence corollary we know there exist a unique solution to the state system also with the control $\bar{u}$. Extracting a further subsequence if necessary still index by $k'$ we have that 

\begin{equation*}
    \theta_{k'} \rightharpoonup \theta \text{ weakly in } W(0,T) \text{ and strongly in } L^2(Q)
\end{equation*}
We want to show that this limit is a solution to the weak form of our state system. We use the weak from as derived in Section 2.1 but now with $\theta_{k'}$ and $u_{k'}$ instead 

Take a test function $\phi \in W(0,T)$ and integrate over our space-time cylinder $Q$ we then have 
\begin{align*}
    \rho c_p \int_0^T\int_{\Omega}\frac{\partial \theta_{k'}}{\partial t}\phi \dxdt + k \int_0^T \int_{\Omega}\nabla \theta_{k'} \cdot \nabla \phi \dxdt \\ + \int_0^T \bigg (\int_{\Gamma_1}\theta_{k'}\phi \ds \bigg )u_{k'}(t) \dt =
    \int_0^T \bigg (\int_{\Gamma_1}\theta_w\phi \ds \bigg ) u_{k'} \dt 
\end{align*}
We want to see what happens if we pass to the limit $k' \rightarrow \infty$ in the equation. All terms here are continuous in $u_{k'}$ and $\theta_{k'}$ we can therefore pass to the limit $k' \rightarrow \infty$. The solution to the state system is unique thus 
\begin{align*}
    \theta(\bar{u}) = \bar{\theta} 
\end{align*}

Now our functional $J$ is lower semicontinous by the previous lemma therefore we have  
\begin{align*}
    J(\bar{\theta}, \bar{u}) \leq \lim \inf_{k'\rightarrow \infty} J(\theta_{k'}, u_{k'})
\end{align*}
therefore the solution is optimal and hence there exist at least one optimal control $\bar{u} \in U_{ad}$. 
\end{proof}