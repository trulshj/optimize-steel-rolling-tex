\section{Existence and uniquenes for the optimal control\\\mbox{problem}}\label{proof}

\subsection{Existence and uniqueness state system}
We will now consider the existence of a solution to the state equation. In order to do so, we consider a more general linear parabolic partial differential equation, show a solution exists for this general form. We then use that result to conclude about the existence for our particular problem as well.

Consider the PDE given by
%
\begin{align}
    \label{eq:moreGen}
    u_t - Au + c_0 y = f \quad \text{in } Q = \Omega \times (0,T) \\
    \label{eq:moreGen2}\partial_{n_A} u + \alpha u = g \quad \text{on } \Sigma = \Gamma \times (0,T) \\
    u(\placeholder,0) = u_0(\placeholder) \quad \text{in } \Omega
\end{align}
It does not predescribe any difficulties by splitting the boundary $\Gamma$ into disjoint parts
and predescribe Neumann conditions on these sections separately, as we have done in \ref{eq:heat}. The functions $\alpha$, $\beta$, $f$ and $g$ all depend on $(x,t)$. Here $A$ is an elliptic differential operator of the following form:
%
\begin{equation*}
    Af(x) \coloneqq -\sum_{i,j}^N\frac{\partial}{\partial x_i}(a_{ij}(x)\frac{\partial}{\partial x_j}f(x)) \quad x\in \Omega
\end{equation*}
One need some requirements on the coefficents of the matrix $A=(a_{ij})$, that is one require $a_{ij} \in L^{\infty}(\Omega)$ and that they are symmetric, i.e. $a_{ji}(x) = a_{ij}(x)$ for $x\in \Omega$. Moreover, one also assumes that the uniform ellipticity condition is satisfied, that is there is some $M > 0$ such that 
\begin{equation*}
    \label{eq:uniformEl}
    \sum_{i,j}^N a_{ij}(x)x_i x_j \geq M|x|^2 \quad \forall x \in \mathbb{R}^N \text{ for a.e. $x\in \Omega$}
\end{equation*}
The operator $\partial_{n_A}$ is the directional derivative of the conormal vector $n_{A}$, a vector whose components are given by $n_{A} = An$, where A is the elliptic differential operator. We present a theorem stating the existence and uniqueness for a solution to \eqref{eq:moreGen}. 

\begin{theorem}[Existence and uniqueness] Assume that $\Omega \subset \mathbb{R}^n$ is a bounded Lipschitz domain with boundary $\Gamma$ and let $T>0$ denote the final time. Moreover, assume $c_0 \in L^{\infty}(Q)$ and $\alpha \in L^{\infty}(\Sigma)$, where $\alpha(x,t) \geq 0$ for a.e. $(x,t) \in \Sigma$, $y_0 \in L^2(\Sigma), f \in L^2(Q)$, and $g \in L^2(\Sigma)$. Then the parabolic initial-value problem \eqref{eq:moreGen} has a unique weak solution in $H^{1,0}(Q) \cap L^{\infty}(0,T;L^2(\Omega))$. Moreover, there is a constant $C>0$ which is independent of $f$, $g$ and $u_0$ such that 
\begin{equation*}
    \max_{t \in [0,T]} \|u(.,t)\|_{L^2(\Omega)} + \|u\|_{W_2^{1,0}(Q)} \leq C\left(\|f\|_{L^2(Q)} + \|g\|_{L^2(\Sigma)} + \|u_0\|_{L^2(\Omega)}\right).
\end{equation*}
This holds for all $f \in L^2(Q), g \in L^2(\Sigma)$ and $u_0 \in L^2(\Omega)$
\end{theorem}

\begin{proof}
The proof is given by Tröltzsch in \cite{optimalControl} under theorem 7.8. We will provide a rough sketch of the required steps in the proof. WOLOG one may assume that $c_0(x,t)\geq 0$ for a.e. $(x,t) \in Q$, since if that were not the case one could do the substitution $y(x,t) \rightarrow e^{\lambda t}\bar{y}(x,t)$. Then the resulting differential equation for $\bar{y}$ involves the term $(\lambda + c_0)\bar{y}$ instead of $c_0y$ as in the original PDE. Taking $\lambda >0$ large enough, this will certainly be positive. 

In order to prove the existence in the theorem one need to proceed in 4 steps: Make a Galerkin approximation to the problem, estimate the sequence $\{y_n \}$ which is an approximation sequence for the states of the PDE, consider the convergence of the sequence of estimated controls $\{u_j^N\}_j$ and states $\{ y_n \}$, and then show that the limit $y_n \rightarrow y$ is indeed a weak solution to the PDE. Finally, to prove uniqueness, one has to use an energy inequality together with the regularity of the solution to complete the proof.
\end{proof}


From this theorem, we can make a conclusion regarding the existence of a solution to our particular parabolic PDE. Let $c_0 = 0$, $A = \frac{1}{\rho c_p}\nabla \cdot (k\nabla)$, and $f = 0$. Then $\partial_{n_A} = \partial_n$. We also set $\Gamma = \Gamma_1 \cup \Gamma_2$, to partition our boundary into to disjoint separate parts, now let $\alpha_1 = -\frac{u(t)}{k}$ and $g_1 = -\frac{u(t)\theta_w}{k}$, while one set $\alpha_2 = 0, g_2 = 0$, where $\alpha_i$ is the value of $\alpha$ on $\Sigma_i$ and similarly for $g_i$. This results in the next corollary. 

\begin{corollary}[Existence]
Suppose that $\theta_w \in L^{\infty}(\Sigma_1)$, $\theta_0 \in L^2(\bar{\Omega})$, $u \in L^{\infty}(0,T)$ and $u\geq 0$. Then the initial value \eqref{eq:heat} admits a unique solution $\theta \in H^{1,0}(Q)$. After a possible modification on a null set, we have $\theta \in W(0,T)$. This weak solution to the PDE satisfy an upper bound 
\begin{equation*}
    \|\theta \|_{W(0,T)} \leq C\bigg ( \|g_1\|_{L^2(\Sigma)} + \|\theta_0\|_{L^2(\Sigma)} \bigg )
\end{equation*}
for a constant $C>0$ indepdentent on $(g_1, \theta_0)$. This can be reformulated as the operator $(g_1,\theta_0) \rightarrow y$ defines a continuous linear operator from $L^2(\Sigma_1)\times L^2(\Sigma)$ into $W(0,T)$; in particular, into $C([0,T];L^2(\Omega))$.  
\end{corollary}

\subsection{Existence and Uniqueness of the Adjoint System}
Our goal is to prove that there exist a unique weak solution $p \in H^{1,0}(Q)$ to our adjoint system. Again, doing a modification of $p$ on a null set, we get $p\in W(0,t)$. We consider the sightly more general parabolic initial-boundary value problem 
\begin{align*}
    -p_t -\nabla^2p +c_0p = a_Q \qquad \text{in } Q, \\
    \partial_np + \alpha p = a_{\Sigma} \qquad \text{on } \Sigma, \\
    p|_{t=T} = a_{\Omega} \qquad \text{in } \Omega.
\end{align*}
If we assume bounded and measurable coefficient functions i.e. $c_0, \alpha, a_Q \in L^2(Q)$, $a_{\Sigma} \in L^2(\Sigma)$, and $a_{\Omega} \in L^2(\Omega)$, then we can introduce a bilinear form
\begin{equation}
    \label{eq:parabolic_adj}
    A(y,v)(t) := \int_{\Omega}(\nabla y \cdot \nabla v + c_0(\placeholder,t) y v) \dx + \int_{\Gamma}\alpha(\placeholder,t) y v \ds
\end{equation}
Then the stated PDE is well-posed according to the following Lemma:

\begin{lemma}[Well-posedness]
The parabolic adjoint system given in \eqref{eq:parabolic_adj} has a unique weak solution $p \in H^{1,0}(Q)$ which is a solution to the variational problem 
\begin{equation*}
    \iint_Q pv_t \dxdt + \int_0^TA(p,v)(t)\dt = \int_{\Omega}a_{\Omega}v(T) \dt + \iint_Q a_Qv \dxdt + \iint_{\Sigma}a_{\Sigma}v \dsdt
\end{equation*}
this should hold for all $v \in H^{1,1}(Q)$ with $v|_{t=0} = 0$. If we modify $p$ on a null set we have $p\in W(0,T)$, and there exists some constant $M>0$ that does not depend on $(a_Q,a_{\Sigma}, a_{\Omega})$ such that 
\begin{equation*}
    \|p\|_{W(0,T)} \leq M \bigg (\|a_Q\|_{L^2(Q)} + \|a_{\Sigma}\|_{L^2(\Sigma)} + \|a_{\Omega}\|_{L^2(\Omega)} \bigg ).
\end{equation*}
\end{lemma}

\begin{proof}
The idea is to reduce the system to a forward parabolic initial-boundary value problem. This is done with a time transformation, taking $\tau \in [0,T]$ and introducing
\begin{equation*}
    \hat{p}(\tau) := p(T-\tau).
\end{equation*}
A similar transform is applied with $\hat{v}(\tau)$. Then $\hat{p}(0) = p(T)$ and $\hat{p}(T) = p(0)$; analogously for $\hat{v}$. We have that $\hat{a_Q}(\placeholder,t):= a_Q(\placeholder,T-\tau)$, and likewise for all the other coefficients. Due to this time transform, we have that
\begin{equation*}
    \iint_Qpv_t \dxdt = - \iint_Q \hat{p}\hat{v_{\tau}} \dxdt.
\end{equation*}
Considering the weak formulation, it now corresponds to the forward parabolic initial-boundary value problem given by 
\begin{align*}
    \hat{p_{\tau}} - \nabla^2 \hat{p} + c_0 \hat{p} = \hat{a_Q} \qquad \text{in } Q \\
    \partial_n \hat{p} + \alpha \hat{p} = \hat{a_{\Sigma}} \qquad \text{on } \Sigma \\
    \hat{p}(0) = \hat{a_{\Omega}} \qquad \text{in } \Omega
\end{align*}
By the previous existence and uniqueness theorem, this problem admits a unique weak solution $\hat{p} \in W(0,T)$. Reversing the time transformation we have proven uniqueness and existence for the adjoint system.
\end{proof}

This system is more general than ours, so we can reduce it to our case. Consider the adjoint equation \eqref{eq:adjoint-eqn}. If we set $c_0 = 0$, $a_Q = 0$ and split up the boundary $\Sigma = \Sigma_1 \cup \Sigma_2$ with $\alpha_1 = \frac{u(t)}{k}$, $\alpha_2 = 0$, and $a_{\Sigma_i}=0$ for $i\in \{1,2 \}$, and then finally set $a_{\Omega} = \theta|_{t=T}-\theta_d$, we have reduced to the case of the lemma. This shows existence and uniqueness for \eqref{eq:adjoint-eqn}.

\subsection{Existence and uniqueness optimal control}
Under the required conditions for a solution to \eqref{eq:heat}, we can go on to prove the existence of an optimal control $\bar{u}$.

\begin{theorem}[Optimal Control]
If the assumptions in the previous theorem are satisfied, then there exist at least one solution to the optimal control problem 
\begin{equation*}
       \min J(\theta, u) = \frac{1}{2} \int_\Omega (\theta(x, T) - \theta_d)^2 \mathop{dx} \mathop{dt} + \frac{\gamma}{2} \int_0^{T} u^2 \mathop{dt}
\end{equation*}
subject to
\begin{align*}
       \rho c_p \theta_t - \nabla \cdot (k \nabla \theta) &= 0 \quad &\text{in } Q,  \\
      -k \frac{\partial \theta}{\partial \nu} &= u(t) (\theta - \theta_w) \quad &\text{on } \Sigma_1, \\
      -k \frac{\partial \theta}{\partial \nu} &= 0 \quad &\text{on } \Sigma_0, \\
      \theta(x, 0) &= \theta_0 & \text{in $\Omega$.}
\end{align*}
\end{theorem}

\begin{proof}
We follow the same footsteps as in \cite{DPSteel}. By the theorem about the existence of a unique solution we know there exists a state $\theta \in W(0,T) \times C([0,T];L^2(\Omega))$ that solves the state equation for every control $u \in U_{ad}$. As $U_{ad}$ is bounded by $L^{\infty}(0,T)$, the solution $\theta$ is bounded in $W(0,T)$. The cost functional is thus bounded, and hence there exists a minimizing sequence $\{(\theta_k,u_k)\}_{k\in \mathbb{N}}$ such that we have 
\begin{equation*}
    \lim_{k \to \infty}J(\theta_k,u_k) = \inf_{(\theta,u)}J(\theta,u).
\end{equation*}

We introduce a solution operator $S(u_k):= (\theta_k, u_k)$, i.e. $(\theta_k,u_k)$ is a solution to the state system corresponding to the control $u_k$. Assuming box-constraints, the set $U_{ad}$ is bounded, closed and convex; thus there exist a subsequence $\{u_{k'} \}_{k'}$ of controls such that one have weak convergence towards a limit in $L^2(0,T)$ such that
\begin{equation*}
    u_{k'} \rightharpoonup \bar{u} \text{ in } L^2(0,T).
\end{equation*}
Using our existence corollary we know there exist a solution to the state system with $u_k$ as our choice of control. We might therefore extract a new subsequence, if necessary, which we still index by $k'$ such that
\begin{equation*}
    \theta_{k'} \rightharpoonup \theta \text{ in } W(0,T) \text{ and strongly in } L^2(Q)
\end{equation*}
Taking a test function $\phi \in H^1(Q)$ and integrating over our space-time cylinder $Q$ we then have 
\begin{align*}
    \rho c_p \int_0^T\int_{\Omega}\frac{\partial \theta_{k'}}{\partial t}\phi \dxdt + k \int_0^T \int_{\Omega}\nabla \theta_{k'} \cdot \nabla \phi \dxdt \\ + \int_0^T \bigg (\int_{\Gamma_1}\theta_{k'}\phi \ds \bigg )u_{k'}(t) \dt =
    \int_0^T \bigg (\int_{\Gamma_1}\theta_w\phi \ds \bigg ) u_{k'} \dt 
\end{align*}
We want to see what happens if we pass to the limit $k' \rightarrow \infty$ in the equation. Due to the continuity of the solution $\theta_{k'}$ we can pass to the limit and all terms converge due to their linearity in $\theta_{k'}$. The solution to the state system is unique, meaning
\begin{align*}
    \theta(\bar{u}) = \bar{\theta} 
\end{align*}
Due to the lower semicontinuity of our functional J we have that 
\begin{align*}
    J(\bar{\theta}, \bar{u}) \leq \lim \inf_{k'} J(\theta_{k'}, u_{k'}),
\end{align*}
and the solution is optimal. 
\end{proof}