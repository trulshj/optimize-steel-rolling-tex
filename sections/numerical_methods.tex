\section{Numerical implementation}
To solve our optimal control problem we will consider two Black-box methods \cite{Algorithms}, and compare their performance. We will implement the simpler Gradient Descent method and compare it to Newton's method. The gradient descent method has an infamous slow convergence rate, compared to the super linear convergence rate of Newton's method. Now since both methods are Black-Box methods, we need to consider the reduced optimal control problem. Let $S$ be the solution operator for our state system. That is, let $S(u):=\theta$ so given a control $u$, $S$ transform the control into the associated temperature distribution i.e. the solution of the state system. Then we can write our cost functional on reduced form, so our objective become to minimize the reduced cost functional 
\begin{equation}
\label{eq:optimization_prob}
    \min_{u \in U}F(u) := J(S(u),u) = \frac{1}{2}\int_{\Omega}(S(u) - \theta_d)^2 \dx+ \frac{\gamma}{2}\int_0^Tu(t)^2\dt
\end{equation}


\subsection{Projected Gradient Descent}
The idea of a descent method consist of finding an optimal direction $d_k$ and stepsize $a_k$ at a given iteration $k$, where say our control is given by $u_k$. We want to find
\begin{equation*}
    F(u_k + \alpha_kd_k) < F(u_k)
\end{equation*}

One Taylor expand the LHS around $u_k$ to get a way to choose the descent direction, typically one can take different means to decide this, but a descent direction can in general be chosen by 
\begin{equation*}
    d_k = \arg\min_{||d||_U=1} \langle \nabla F(u_k), d \rangle_U
\end{equation*}
We consider gradient descent therefore we choose the descent direction by setting $d_k = -\nabla F(u_k)$. After deciding the gradient direction one need to determine how far to go in that direction, this is decided by $\alpha_k$, called \textit{line search} parameter. The ideal such parameter is 
\begin{equation*}
    a_k = \arg \min_{\alpha>0} \{ F(u_k + \alpha d_k) \}
\end{equation*}
There are different ways to choose this line search step. One require a couple of properties to be satisfied by this line search parameter these are
\begin{align*}
    F(u_k + \alpha_kd_k) < F(u_k) \text{  } \forall k \\
    F(uk + \alpha_k d_k) - F(u_k) \rightarrow 0 \text{ as } k\rightarrow \infty
\end{align*}
We stick with Armijo rule as our line search strategy. This method consists of given a descent direction $d_k$ of F at $u_k$ choose the largest $a_k \in \{ \beta^0, \beta, \beta^2,..,\beta^m,... \}$ such that
\begin{equation*}
    F (u_k + \alpha_kd_k) - F(u_k) \leq \mu \alpha_k \langle \nabla F(u_k),d_k \rangle_{U}
\end{equation*}
where $\mu,\beta  \in (0,1)$ are chosen constant, that is we use backtracking to determine the stepsize \cite{iterativeMethods}. We set $\mu = 10^{-3}$ and $\beta = \frac{1}{2}$. More facts about Armijo rule and the convergence properties of this line search strategy is in \cite{numMethods} \cite{iterativeMethods}. However we have control-constraints, so we need to ensure the argument is placed inside the admissable set of controls $U_{ad}$, in particular we have box constraints given in \eqref{eq:box_constraints}, that is we have a lower bound $u_a$ and upper bound $u_b$ for the control variable. Therefore we need to use a projection onto the set of admissable controls, that is why it is called the projected gradient descent method. We introduce a cut-off function i.e. a projection given by
\begin{equation}
    \label{eq:projection}
    \mathbb{P}_{[u_a,u_b]}(v) := \max \{u_a, \min \{u_b,v \} \}
\end{equation}
this should hold componentwise, if one allow $u_a$ and $u_b$ to be vectors, notation is similar to \cite{Algorithms}. One have to introduce this into the gradient descent method and Armijo's rule. That is we have to consider the projected Armijo's rule which is determine $\alpha_k \in \{1, \frac{1}{2},\frac{1}{4},.. \}$ such that 
\begin{equation*}
    F \bigg (\mathbb{P}_{[u_a,u_b]}(u_k + \alpha_kd_k) \bigg ) - F(u_k) \leq \mu \alpha_k \langle \nabla F(u_k),d_k \rangle_{U}
\end{equation*}
A pseudocode for the whole Projected Gradient Descent method is shown in Algorithm 1. We introduce two stopping criterion's, one based on the minimum allowed step length, and one on the difference in the computed values of the control given by 
\begin{equation}
    \label{eq:stopping}
    \tau := ||u^{(k+1)} - u^{(k)}||_U 
\end{equation}
This is to check that the controls converge towards a value, $U$ is the space of the controls. We therefore want to consider an optimization algorithm which is based on a quadratic approximation. 

\begin{codebox}
\Procname{$\proc{Algorithm1: Projected Steepest Descent with Armijo}$}
\li Choose initial value $u^{(0)}\in U_{ad}$
\li Set tolerance $\id{tol}$ and $\alpha_{min}$
\li Solve state system to obtain $\theta^{(0)}$, with $u^{(0)}$
\li Solve adjoint system to obtain $p^{(0)}$, with $u^{(0)}$ and $p^{(0)}$
\li $k=0$
\li \While ($\tau > \id{tol}$)  \Then
\li Choose the descent Direction $d_k = -\nabla F(u^{(k)})$
\li Use the Projected Armijo rule to determine $\alpha_k$
\li \If $\alpha_k < \alpha_{min}$ \Then 
\li \textbf{Break} \End
\li Set $u^{(k+1)} = \mathbb{P}_{[u_a,u_b]}\bigg (u^{(k)} + \alpha_k d_k \bigg )$
\li Solve state system to obtain $\theta^{(k+1)}$, with $u^{(k+1)}$
\li Solve adjoint system to obtain $p^{(k+1)}$, with $u^{(k+1)}$ and $\theta^{(k+1)}$
\li Compute $\tau$
\li $k = k +1$
\end{codebox}
This is the projected descent algorithm which we will apply to our control problem. The gradient of our reduced cost functional is given by

\begin{equation*}
    \nabla F(u) =  \gamma u - \frac{1}{\rho c_p} \int_{\Gamma_1} (\theta - \theta_w)p \ds = 0
\end{equation*}

The projected gradient descent method is globally convergent if the reduced cost functional F is Fréchet differentiable and bounded from below. Now since the solution to the state system $S(u) \in W(0,T) \cap C(\bar{Q})$, $F$ is indeed Fréchet differentiable. Moreover we have that $F(u) \geq 0$ for all choices of controls $u \in U_{ad}$ by the definition of $F(u)$ therefore we have global convergence. However this is a first-order descent method, and the convergence rate is typically very slow, since it solely rely on first-order gradient information. We expect to see pretty fast convergence when the control approximated thus far $u_k$ is far from the optimal, and the convergence gets slower and slower once we get closer to the optimal $\bar{u}$ 

\subsection{Newton's method}
Another type of direction is obtained at each direction $d_k$ if one instead use a quadratic expansion. For Newton's method we need again the reduced cost function $F$ to be Fréchet differentiable and that its Fréchet derivative is boundedly invertible $\forall u \in U_{ad}$, or at least close to the true solution $\bar{u}$. We try to approximate a solution of the equation $F(u) = 0$ by repeatedly linearizing the function and solving the linearized system
\begin{align}
    \label{eq:Newton}
    u_{k+1} = u_k - F'(u_k)^{-1}F(u_k)
\end{align}
Want to use Newton's method with a primal-dual active set strategy to handle the control constraint. In the unconstrained control case for an optimal control problem, a necessary optimality condition for the reduced control problem is that $\nabla F(u)$ vanishes. However in the presence of control constraints, this condition must be replaced by the variational inequality
\begin{equation}
    \langle \nabla F(\bar{u}), (v-\bar{u}) \rangle \geq 0 \text{ } \forall v \in U_{ad}
\end{equation}
We can reformulate this using Lagrange multipliers, requiring 
\begin{equation*}
    \nabla F(u) + \mu_b - \mu_a =0 
\end{equation*}
to get the complementary slackness conditions which are given by the following equations 
\begin{align}
    \label{eq:complementary}
    \mu_a \geq 0, \quad u_a - u \leq 0, \quad \mu_a(u_a - u) = 0 \\
    \mu_b \geq 0, \quad u - u_b \leq 0, \quad \mu_b(u-u_b) =0
\end{align}
To treat \eqref{eq:complementary} numerically, one combine this into one equation, by introducing a new Lagrange multiplier $\mu = \mu_b - \mu_a$, one convert the complementary slackness conditions into one equation involving maximum and minimum. Therefore the system become 
\begin{align}
    \label{eq:finalComp}
    \nabla F(u) + \mu = 0 \\
    \max \{0, \mu + c(u-u_b) \} + \min \{0, \mu + c(u-u_b) \} - \mu = 0
\end{align}

The latter equation is just a reformulation of the two equations in \eqref{eq:complementary}. Newton's method is based on calculating the Hessian of the reduced cost functional, since it uses second order information about the functional. This depends on the primal variable $u$ and the dual variable $\mu$ we introduce the active and inactive sets at an iterate $(u_k, \mu_k)$ by 
\begin{align}
    \label{eq:active_inactive}
    A_k^{+} := \{ t \in [0,T]: \text{ } \mu_k + c(u_k - u_b) >0 \} \\
    A_k^{-} := \{ t \in [0,T]: \text{ } \mu_k + c(u_k - u_a) <0 \} \\
    I_k := [0,T] \backslash A_k \quad A_k := A_k^{+} \cup A_k^{-}
\end{align}

The Newton direction at an iterate $(u_k, \mu_k)$ can be calculated by solving the following symmetric form of the Newton system

\begin{equation}
    \label{eq:Newton_system}
    \begin{bmatrix}
        \nabla^2 F(u_k) & \chi_{A_k} \\
        \chi_{A_k} & 0 
    \end{bmatrix}
    \begin{bmatrix}
    \delta u \\
    \delta \mu|_{A_k}
    \end{bmatrix}
    = - \begin{bmatrix}
    \nabla F(u_k) + \mu_k \\
    \chi_{A_k^{+}}(u_k - u_b) + \chi_{A_k^{-}}(u_k - u_a)
    \end{bmatrix}
\end{equation}
The next dual variable $\mu_{k+1}$ is set to be zero on the the inactive set $I_k$. A Pseudocode for Newton's method with primal-dual active set strategy is given in Algorithm 2

\begin{codebox}
\Procname{$\proc{Algorithm2: Newton's method with Primal-Dual Active Set Strategy}$}
\li Choose initial value $u^{(0)} \in U_{ad}$ and $\mu_0$ 
\li Set tolerance $tol$
\li Set $k = 0$
\li Solve state system to obtain $\theta^{(0)}$
\li Solve adjoint system to obtain $p^{(0)}$
\li \While $(\tau > tol)$ \Then 
\li Evaluate the reduced gradient $\nabla F(u_k)$
\li Compute the active sets $A_k^{+}$ and $A_k^{-}$
\li \If $A_k^{+} = A_{k-1}^{+}$ and $A_{k-1}^{-} = A_k^{-}$ \Then 
\li \textbf{Break} \End
\li Solve the Newtonian system given in \eqref{eq:Newton_system} iteratively
\li Set $u^{(k+1)} = u^{(k)} + \delta u$  
\li Set 
\begin{equation*}
    \mu_{k+1}(t) = 
    \begin{cases}
     u_k + \delta u \text{ if } t \in A_k \\
     0 \qquad \text{ if } t \in I_k
     \end{cases}
\end{equation*}
\li Solve state system to obtain $\theta(^{(k+1)}$ using $u^{(k+1)}$
\li Solve adjoint system to obtain $p^{(k+1)}$ 
\li Compute $\tau$
\li Set $k = k+1$
\end{codebox}

Where $\tau$ is calculated as in Algorithm 1. We do not globalise the algorithm here, but that can be done to achieve convergence from an arbitrary starting point. This is done by for instance carrying out a few gradient descent steps in the starting phase, and then switching over to Newton's method. The notion used here is similar to that in \cite{Algorithms}. 

Expect to observe superior convergence properties of Newton's method compared to Steepest Descent, as Newton's method is superlinearly convergent compared to $q$-linearly convergent. The rate of convergence can be measured similarly as in \cite{DPSteel} by setting 
\begin{equation}
    \label{eq:rate_of_conv}
    e_k := \frac{||u^{(k)}-\bar{u}||_U + ||\theta^{(k)}-\bar{\theta}||_{L^2(Q)} +||p^{(k)}-\bar{p}||_{L^2(Q)} }{||u^{(k+1)}-\bar{u}||_U + ||\theta^{(k+1)}-\bar{\theta}||_{L^2(Q)} +||p^{(k+1)}-\bar{p}||_{L^2(Q)}}
\end{equation}
where $(\bar{\theta},\bar{u},\bar{p})$ is the analytical solution to the control problem. We use this as a convergence test of the two algorithm on a control problem where we can find an analytical solution. 

Other mentionable methods to use on the optimal control problem is the sequential quadratic programming (SQP) method \cite{Algorithms}, and reduced SQP \cite{DPSteel}. 
