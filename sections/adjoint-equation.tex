
In this section we will use the formal Langrange method to derive the adjoint equation and the optimality system \cite{optimalControl}. Formal in the sense that we use a lagrange multiplier $p$ without proving its existence in the derivation done her, we do that in Section \ref{proof}. The state system for our optimal control problem is given by a linear parabolic partial differential equation, in particular a heat equation

\begin{align*}
      \rho c_p \theta_t - \nabla \cdot (k \nabla \theta) &= 0 \quad &\text{in } \Omega \\
      -k \frac{\partial \theta}{\partial \nu} &= u(t) (\theta - \theta_w) \quad &\text{on } \Gamma_1, \\
      -k \frac{\partial \theta}{\partial \nu} &= 0 \quad &\text{on } \Gamma_0, \\
      \theta(x, 0) &= \theta_0 &
\end{align*}
we can draw the domain $Q = \Omega \times (0,T)$, for the hot rolling of steel, we can draw our domain as a steel slab shown in .... More generally we require that $\Omega \subset \mathbb{R}^3$ is a bounded domain with a Lipschitz boundary, $\partial \Omega = \Gamma_1 \cup \Gamma_2$. A suitable function space for the solution of a linear parabolic PDE is 
\begin{equation}
    \label{eq:funcSpace}
    W(0,T) : = \{ \theta \in L^2(0,T;H^1(\Omega)) : \quad \frac{\partial \theta}{\partial t} \in L^2(0,T;(H^1(\Omega))^{*}) \}
\end{equation}
It is natural to endow this space with the following norm, let $u \in W(0,T)$ then 
\begin{equation*}
    \|u\|_{W(0,T)} := \bigg (\int_{0}^T\|u(t)\|^2_{H^{1}(\Omega)}\dt \bigg ) ^{\frac{1}{2}} + \bigg (\int_{0}^T\|u'(t)\|^2_{H^{1}(\Omega)^{*}}\dt \bigg ) ^{\frac{1}{2}}
\end{equation*}
In general we mean by the notation $L^{p}(a,b,X)$ the linear space of all equivalence classes of measurable vector valued functions $u:[a,b] \rightarrow X$ which have the property that
\begin{equation*}
    \int_{a}^b\|u\|_X^p \dt<\infty
\end{equation*}

\subsection{The adjoint eqution and the gradient}
We start by finding the weak formulation for the problem defined in \eqref{eq:heat}. We multiply the equation \eqref{eq:heat-in-omega} by a test function $\psi\in H^{1,1}(Q)$. That is a function which has weak first-order partial derivatives in space and time, and integrate over the domain $Q$. We furthermore assume that the heat conductivity, $k$, heat capacity $c_p$ and density $\rho$ are all scalars, partial integration in space then yields 
\begin{equation}
\begin{aligned}
  0 &= \iint_Q (\theta_t - \frac{k}{\rho c_p}\Delta\theta)\psi\dxdt  \\
  &= \iint_Q \theta_t\psi\dxdt + \frac{k}{\rho c_p}\iint_Q\nabla\theta \cdot \nabla\psi\dxdt - \frac{k}{\rho c_p}\iint_{\partial Q}\partial_\nu\theta\psi\dsdt.
\end{aligned}
\end{equation}
Now after inserting for the boundary conditions defined in \eqref{eq:heat}, we end up with the following weak formulation of the state equation
\begin{align}\label{eq:weak-form}
  \iint_Q \theta_t\psi\dxdt + \frac{k}{\rho c_p}\iint_Q\nabla\theta \cdot \nabla\psi\dxdt + \frac{1}{\rho c_p}\iint_{\Sigma_1} u(t)(\theta - \theta_w)\psi\dsdt = 0\\
  \theta|_{t=0} = \theta_0
\end{align}
where $\Sigma_i = \Gamma_i\times(0,T)$. We view $\theta_t$ as being in the space $ L^(0,T;H^1(\Omega)^{*})$. Due to the isolation Neumann condition at $\Sigma_2$ all these boundary terms vanishes. Now one can use a density argument to lessen the requirement on the test function such that one only require that $v \in W(0,T)$ which is introduced below. In fact this weak formulation is even valid for $v \in L^2(0,T,H^1(\Omega))$. 

From the weak formulation of the problem, we can easily set up the Lagrangian function, by subtracting the weak formulation of the problem \eqref{eq:weak-form} from the cost functional \eqref{eq:cost-func}. The test function is replaced with a Lagrange multiplier function $p$. This gives
\begin{equation}
  \begin{aligned}\label{eq:lagrangian-raw}
  \L(\theta, u, p) = &\,J(\theta, u) - \iint_Q \theta_t p\dxdt - \frac{k}{\rho c_p}\iint_Q\nabla\theta \cdot \nabla p \dxdt \\
  &- \frac{1}{\rho c_p}\iint_{\Sigma_1} u(t)(\theta - \theta_w)p \dsdt
  \end{aligned}
\end{equation}
As we can see our Lagrangian $\L : W(0,T)\cap C(\bar{Q}) \times L^{\infty}(0,T) \times  W(0,T) \cap C(\bar{Q}) \rightarrow \mathbb{R}$.

\subsubsection{Adjoint equation}
Now we derive the adjoint equation. This is done by  setting the directional derivative of the Lagrangian (with respect to the state variable) equal to zero. Assume the direction $h\in H^1(Q)$ is such that $h(x, 0) = 0$. % Antagelsen h(x, 0) = 0 popper visst ut av seg selv hvis man gjør noe på en bestemt måte i følge Dietmar.
This gives
\begin{equation}
  \begin{aligned}
  0 = \L_\theta(\theta, u, p)h = \int_\Omega (\theta(x,T) - \theta_d)h(x, T)\dx - \iint_Q h_t p\dxdt \\
  - \frac{k}{\rho c_p}\iint_Q\nabla h\nabla p \dxdt
  - \frac{1}{\rho c_p}\iint_{\Sigma_1} u(t)h p\dsdt.
  \end{aligned}
\end{equation}
Partial integration in time and in space then yields
\begin{equation} % Skip this calculation?
  \begin{aligned}
  0 = \L_\theta(\theta, u, p)h = \int_\Omega \theta(x,T) - \theta_d)h(x, T)\dx - \int_\Omega hp\Big|_0^T \dx \\
  + \iint_Q h p_t\dxdt
  + \frac{k}{\rho c_p}\iint_Q h\Delta p \dxdt \\
  - \frac{k}{\rho c_p} \iint_{\partial Q}\partial_\nu p\cdot h\dsdt
  - \frac{1}{\rho c_p}\iint_{\Sigma_1} u(t)h p\dsdt.
  \end{aligned}
\end{equation}
Now after som reordering of the terms we get
\begin{equation}
  \begin{aligned}
  0 = \L_\theta(\theta, u, p)h = \int_\Omega \theta(x,T) - \theta_d-p(x, T))h(x, T)\dx \\
  + \iint_Q h \left( p_t + \frac{k}{\rho c_p}\Delta p\right) \dxdt
   - \frac{k}{\rho c_p} \iint_{\Sigma_2}\partial_\nu p\cdot h\dsdt \\
   - \frac{1}{\rho c_p} \iint_{\Sigma_1}h(  k\partial_\nu p + u(t)p )\dsdt.
  \end{aligned}
\end{equation}
Now we let $h\in H_0^1(Q)$ i.e. vanishing at the boundary. Then we are only left with the second integral, which must still be equal to zero for every $h$ in the chosen function space. We then get that
\begin{equation*}
  \rho c_p p_t + k\Delta p = 0 \quad\textrm{ in } \Omega.
\end{equation*}
Now letting $h\in H^1(Q)$ such that $h(x, T) = 0$ and $h|_{\Sigma_2}=0$ gives that
\begin{equation*}
  -k\frac{\partial p}{\partial\nu} = u(t)p \quad\textrm{ in } \Sigma_1.
\end{equation*}
Now we replace the last condition on $h$ from above with $h|_{\Sigma_1}=0$ and get
\begin{equation*}
  -k\frac{\partial p}{\partial\nu} = 0 \quad\textrm{ in } \Sigma_2.
\end{equation*}
Now finally we just let $h\in H^1(Q)$ and we are left with
\begin{equation*}
  p(x, T) = \theta(x, T) - \theta_d
\end{equation*}
This constitues the \textit{adjoint equation}, which we restate below
\begin{subequations}
   \label{eq:adjoint-eqn}
   \begin{align} % Approved by Dietmar!
      \rho c_p p_t + k\Delta p &= 0 \quad\qquad\textrm{ in } \Omega \time (0,T) \\
      -k\frac{\partial p}{\partial\nu} &= u(t)p \,\,\quad\textrm{ in } \Sigma_1 \\
      -k\frac{\partial p}{\partial\nu} &= 0 \,\quad\qquad\textrm{ in } \Sigma_2 \\
      p(x, T) &= \theta(x, T) - \theta_d. \quad \textrm{ in } \Omega
   \end{align}
\end{subequations}
This adjoint equation for the adjoint state $p$ runs backwards in time, but since the final condition i.e. $p(x,T)$ is predescribed instead of the initial condition, the problem is well-posed. If one had posed $p|_{t=0}$ instead one would have gotten an ill-posed backward parabolic equation. The driving force is end-temperature difference between the desired state and the actual temperature state of the steel slab.

\subsubsection{Gradient}
We assume box-constraints \eqref{eq:box_constraints}, then we can derive the variational inequality which is a first-order neccessary optimality condition to be satisfied by an optimal control $\bar{u} \in U_{ad}$. The variational inequality is given by 
\begin{equation*}
    \L_u(\bar{\theta},\bar{u},p)(u-\bar{u}) \geq 0 \text{ for } \forall u\in U_{ad}
\end{equation*}
Let $h = u - \bar{u}$, we differentiate in direction $h$, but now note that $h\in H^1(0, T)$. This gives
\begin{equation}
\begin{aligned} % Fikk hjelp av Dietmar for denne, så den skal være good.
  \L_u(\theta, u, p) h &= \gamma\int_0^T uh \dt - \frac{1}{\rho c_p} \iint\limits_{0\,\,\Gamma_1}^{\,\,\,T}(\theta - \theta_w)ph \dsdt \\
  &= \int_0^T h \left( \gamma u - \frac{1}{\rho c_p} \int_{\Gamma_1}(\theta - \theta_w)p \mathop{ds} \right) \dt.
\end{aligned}
\end{equation}
There inserting back for $h$ we have derived the variational inequality for our system, which states that
\begin{equation}
    \label{eq:variational}
    \int_0^T (u - \bar{u})(t) \left( \gamma \bar{u} - \frac{1}{\rho c_p} \int_{\Gamma_1}(\bar{\theta} - \theta_w)p \mathop{ds} \right) \dt \geq 0 \text{ for } \forall u \in U_{ad}
\end{equation}

The gradient of the reduced cost funtional $F(u) := J(y(u),u)$ can be obtained from 
\begin{equation*}
    F'(u) = \L_u(y(u),u,p(u))
\end{equation*}

The space $H^1(0,T)$ is a hilbert space. By Riesz representation theorem we can identify $H$ with $H^{*}$ for any Hilbert space $H$. Therefore by identifying using an isomorphism we find that the u-derivative of the lagranginan and consequentely the gradient of our reduced cost functional is

\begin{equation*}
    F'(u) = \L_u(\theta,u,p) = \gamma u - \frac{1}{\rho c_p}\int_{\Gamma_1}(\theta - \theta_w)p \ds.
\end{equation*}


\subsection{Optimality System}
Now the optimality system of our control problem, \eqref{eq:heat} consists of the state equation, the adjoint equation and the variational inequality, the two latter are obtained by requiring that the $y$- and $u$-derivative of the Lagrange function do vanish. Therefore the total optimality system become

\begin{align*}
    \begin{cases}
     \rho c_p \theta_t - \nabla \cdot (k \nabla \theta) = 0 \quad & \text{in $\Omega$}, \\
      -k \frac{\partial \theta}{\partial \nu} = u(t) (\theta - \theta_w) &\text{on } \Gamma_1, \\
      -k \frac{\partial \theta}{\partial \nu} = 0  &\text{on } \Gamma_0, \\
      \theta(x, 0) = \theta_0 
      \end{cases}
      \end{align*}
      \begin{align*}
      \begin{cases}
       \rho c_p p_t + k\Delta p &= 0 \quad\qquad\textrm{ in } \Omega \\
      -k\frac{\partial p}{\partial\nu} &= u(t)p \,\,\quad\textrm{ in } \Sigma_1 \\
      -k\frac{\partial p}{\partial\nu} &= 0 \,\quad\qquad\textrm{ in } \Sigma_2 \\
      p(x, T) &= \theta(x, T) - \theta_d.
      \end{cases}
      \end{align*}
\begin{equation*}
      \gamma u - \frac{1}{\rho c_p} \int_{\Gamma_1} (\theta - \theta_w)p \ds = 0
\end{equation*}

This system is a first order necessary optimality condition. That is if $\bar{u} \in U_{ad}$ is an optimal control to our optimal control problem and $\bar{\theta} = S(\bar{u})$ is the associated solution to the state system. Then $\exists$ an adjoint state $\bar{p}$ such that the adjoin system is satisfied, and the variational inequality is satisfied. 

Now one can restate this optimalilty system using the projection formula if one assume box-constraints. That is if we assume

\begin{equation}
    \label{eq:box_constraints}
    U_{ad} := \{ u \in L^2(\Sigma_1): u_a(x,t) \leq u(x,t) \leq u_b(x,t) \text{ for a.e. } (x,t) \in \Sigma_1 \}
\end{equation}
Then a control $\bar{u} \in U_{ad}$ and the associated state $\bar{y}$ is optimal if and only if it satisfy togheter with the adjoint state $p$ solving \eqref{eq:adjoint-eqn}  and $\gamma >0$ i.e. the regularizaing parameter is positive that
%\begin{equation}
%    \label{eq:Proj_formula}
%    \bar{u(x,t)} = \mathbb{P}_{[u_a(x,t),u_b(x,t)]} \{\frac{1}{\gamma}\theta_w(x,t)p(x,t) \}
%\end{equation}